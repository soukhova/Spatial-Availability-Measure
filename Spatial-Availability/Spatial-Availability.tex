\documentclass[]{elsarticle} %review=doublespace preprint=single 5p=2 column
%%% Begin My package additions %%%%%%%%%%%%%%%%%%%
\usepackage[hyphens]{url}

  \journal{Journal of Transport Geography} % Sets Journal name


\usepackage{lineno} % add

\usepackage{graphicx}
%%%%%%%%%%%%%%%% end my additions to header

\usepackage[T1]{fontenc}
\usepackage{lmodern}
\usepackage{amssymb,amsmath}
\usepackage{ifxetex,ifluatex}
\usepackage{fixltx2e} % provides \textsubscript
% use upquote if available, for straight quotes in verbatim environments
\IfFileExists{upquote.sty}{\usepackage{upquote}}{}
\ifnum 0\ifxetex 1\fi\ifluatex 1\fi=0 % if pdftex
  \usepackage[utf8]{inputenc}
\else % if luatex or xelatex
  \usepackage{fontspec}
  \ifxetex
    \usepackage{xltxtra,xunicode}
  \fi
  \defaultfontfeatures{Mapping=tex-text,Scale=MatchLowercase}
  \newcommand{\euro}{€}
\fi
% use microtype if available
\IfFileExists{microtype.sty}{\usepackage{microtype}}{}
\bibliographystyle{elsarticle-harv}
\ifxetex
  \usepackage[setpagesize=false, % page size defined by xetex
              unicode=false, % unicode breaks when used with xetex
              xetex]{hyperref}
\else
  \usepackage[unicode=true]{hyperref}
\fi
\hypersetup{breaklinks=true,
            bookmarks=true,
            pdfauthor={},
            pdftitle={Estimating spatial availability/mismatch using singly constrained accessibility measures},
            colorlinks=false,
            urlcolor=blue,
            linkcolor=magenta,
            pdfborder={0 0 0}}
\urlstyle{same}  % don't use monospace font for urls

\setcounter{secnumdepth}{0}
% Pandoc toggle for numbering sections (defaults to be off)
\setcounter{secnumdepth}{0}


% tightlist command for lists without linebreak
\providecommand{\tightlist}{%
  \setlength{\itemsep}{0pt}\setlength{\parskip}{0pt}}


% Pandoc citation processing
\newlength{\cslhangindent}
\setlength{\cslhangindent}{1.5em}
\newlength{\csllabelwidth}
\setlength{\csllabelwidth}{3em}
\newlength{\cslentryspacingunit} % times entry-spacing
\setlength{\cslentryspacingunit}{\parskip}
% for Pandoc 2.8 to 2.10.1
\newenvironment{cslreferences}%
  {}%
  {\par}
% For Pandoc 2.11+
\newenvironment{CSLReferences}[2] % #1 hanging-ident, #2 entry spacing
 {% don't indent paragraphs
  \setlength{\parindent}{0pt}
  % turn on hanging indent if param 1 is 1
  \ifodd #1
  \let\oldpar\par
  \def\par{\hangindent=\cslhangindent\oldpar}
  \fi
  % set entry spacing
  \setlength{\parskip}{#2\cslentryspacingunit}
 }%
 {}
\usepackage{calc}
\newcommand{\CSLBlock}[1]{#1\hfill\break}
\newcommand{\CSLLeftMargin}[1]{\parbox[t]{\csllabelwidth}{#1}}
\newcommand{\CSLRightInline}[1]{\parbox[t]{\linewidth - \csllabelwidth}{#1}\break}
\newcommand{\CSLIndent}[1]{\hspace{\cslhangindent}#1}

\usepackage{booktabs}
\usepackage{longtable}
\usepackage{array}
\usepackage{multirow}
\usepackage{wrapfig}
\usepackage{float}
\usepackage{colortbl}
\usepackage{pdflscape}
\usepackage{tabu}
\usepackage{threeparttable}
\usepackage{threeparttablex}
\usepackage[normalem]{ulem}
\usepackage{makecell}
\usepackage{xcolor}



\begin{document}


\begin{frontmatter}

  \title{Estimating spatial availability/mismatch using singly
constrained accessibility measures}
    \author[Some School]{Author One}
   \ead{author.1@example.com} 
    \author[Some School]{Author Two\corref{Corresponding Author}}
   \ead{author.2@example.com} 
      \address[Some School]{Address}
      \cortext[1]{Corresponding Author}
  
  \begin{abstract}
  This is the abstract.

  It consists of two paragraphs.
  \end{abstract}
  
 \end{frontmatter}

\newpage

\hypertarget{introduction}{%
\section{Introduction}\label{introduction}}

The concept of accessibility is a relatively simple one whose appeal
derives from combining the spatial distribution of opportunities and the
cost of reaching them (Hansen, 1959). Numerous methods for calculating
accessibility have been proposed that can be broadly organized into
infrastructure-, place-, person-, and utility-based measures (Geurs and
van Wee, 2004). Of these, the place-based family of measures is arguably
the most common, capturing the number of opportunities reachable from an
origin using the transportation network. This type of measure is also
referred to as a gravity-based measure of accessibility that captures
the potential for interaction.

Accessibility analysis is widely employed in transportation, geography,
public health, and many other areas, and there is increasing emphasis on
a shift from mobility-oriented to accessibility-oriented planning
(Deboosere et al., 2018; Handy, 2020; Proffitt et al., 2017; Yan, 2021).
However, while these types of accessibility measures are excellent
indicators of the intersection between urban structure and
transportation infrastructure, they have been criticized in the past for
not being highly interpretable. Previous research has highlighted how
the weighting of opportunities using an impedance function can make
gravity measures more difficult for planners and policymakers to
interpret compared to simpler cumulative opportunity measures (Geurs and
van Wee, 2004; Miller, 2018). Moreover, because place-based
accessibility measures sensitive to the number of opportunities and the
characteristics of the transportation network, raw accessibility values
cannot be easily compared across study areas (Allen and Farber, 2019).

Intra- and inter-regional comparisons are challenging because
gravity-based accessibility indicators are spatially smoothed estimates
of the total number of opportunities, however, the meaning of their
magnitudes is unclear. This is evident when we consider the ``total
accessibility'' in the region, a quantity that is not particularly
meaningful since it is not constrained to resemble, let alone match the
number of opportunities available. Furthermore, while accessibility
depends on the supply of destination opportunities weighted by the
travel costs associated with reaching them, the calculated
accessibilities are not sensitive to the demand for those opportunities
at the origins. Put another way, traditional measures of place-based
accessibility do not capture the competition for opportunities. This
theoretical shortcoming (Geurs and van Wee, 2004) is particularly
problematic when those opportunities are ``non-divisible'' in the sense
that, once they have been taken by someone, are no longer available to
other members of the population. Examples of indivisible opportunities
include jobs (when a person takes up a job, the same job cannot be taken
by someone else) and placements at schools (once a student takes a seat
at a school, that particular opportunity is no longer there for another
student). From a different perspective, employers may see workers as
opportunities, so when a worker takes a job, this particular individual
is no longer in the available pool of candidates for hiring.

To remedy these issues, researchers have proposed several different
approaches for calculating competitive accessibility measures. On the
one hand, this includes several approaches that first normalize the
number of opportunities available at a destination by the demand for
them from the origin zones and, second, sum the demand-corrected
opportunities reachable from the origins (e.g. Joseph and Bantock, 1984;
Shen, 1998). These advances were popularized in the family of two-step
floating catchment area methods (Luo and Wang, 2003) that have found
widespread adoption for calculating competitive accessibility to
healthcare and other uses. In principle floating catchment areas purport
to account for competition/congestion effects, although in practice
several researchers (e.g., Delamater, 2013; Wan et al., 2012) have found
that they tend to over-estimate the level of demand and/or service. The
underlying issue, as demonstrated by Paez et al. (2019), is the multiple
counting of both population and level of service, which can lead to
biased estimates if not corrected.

A second approach is to impose constraints on the gravity model to
ensure flows between zones are equal to the observed totals. Based on
Wilson's (1971) entropy-derived gravity model, researchers can
incorporate constraints to ensure that the modelled flows match some
known quantities in the data inputs. In this way, models can be
singly-constrained to match the row- or column-marginals (the trips
produced or attracted, respectively), whereas a doubly-constrained model
is designed to match both marginals. Allen and Farber (2019) recently
incorporated a version of the doubly-constrained gravity model within
the floating catchment area approach to calculate competitive
accessibility to employment using transit across eight cities in Canada.
But while such a model can account for competition, the mutual
dependence of the balancing factors in a doubly-constrained model means
they must be iteratively calculated which makes them more
computationally-intensive. Furthermore, the double constraint means that
the sum of opportunity-seekers and the sum of opportunities must match,
which is not necessarily true in every case (e.g., there might be more
people searching for work than jobs exist in a region).

In this paper we propose an alternative approach to measuring
competitive accessibility. We call it a measure of \emph{spatial
availability} (SA), and it aims to capture the number of indivisible
opportunities that are not only \emph{accessible} but also
\emph{available} to the opportunity-seeking population, in the sense
that they have not been claimed by a competing seeker of the
opportunity. As we will show, spatial availability is a
singly-constrained measure of accessibility. By allocating opportunities
in a proportional way based on demand and distance, this method avoids
the issues of inflation that result from multiple counting of
opportunities in traditional accessibility measures. The method returns
meaningful accessibilities that correspond to the rate of available
opportunities per person. Moreover, the method also returns a benchmark
value for the region under study against which results for individual
origins can be compared.

In the following sections we will describe and illustrate this new
measure using simple numerical examples. First, we will describe the
measure. Second, we will calculate the SA using a simple hypothetical
population and employment centers data set for three use-cases: one of
jobs from the perspective of the population, another of workers from the
perspective of employers, and another considering catchment
restrictions. Thirdly, we calculate the SA using real world data for the
Transportation Tomorrow Survey (TTS) home-to-work commute in 2016 for
the Greater Golden Horseshoe (GGH) area in Ontario, Canada. Finally, we
discuss the differences between accessibility estimates to the proposed
measure of SA and the potential range of uses of the SA measure.

\hypertarget{background}{%
\section{Background}\label{background}}

Most accessibility measures (excluding utility-based measures) are
derived from the gravity model, and are known as \emph{gravity-based}
accessibility. Briefly, consider the following accessibility measure
\(A_i\) :

\[
A_i = \sum_{j=1}^JO_jf(c_{ij})
\]

\noindent where:

\begin{itemize}
\tightlist
\item
  \(i\) is a set of origin locations.
\item
  \(j\) is a set of destination locations.
\item
  \(O_j\) is the number of opportunities at location \(j\). These are
  opportunities for activity and add some sort of \emph{supply} to the
  area;
\item
  \(c_{ij}\) is a measure of the cost of moving between \(i\) and \(j\)
\item
  \(f(\cdot)\) is an impedance (or distance-decay) function (a
  monotonically non-increasing or decreasing function of \(c_{ij}\)).
\end{itemize}

The accessibility value \(A_i\), it can be seen, is the weighted sum of
opportunities that can be reached from location \(i\), given the cost of
travel \(c_{ij}\) and an impedance function. Summing the opportunities
in the neighborhood of \(i\) (the neighborhood is defined by the
impedance function) estimates of the total number of opportunities that
can be reached from \(i\) at a certain cost. Depending on the impedance
function, the measure could be cumulative opportunities (if \(f(\cdot)\)
is a binary or indicator function) or a more traditional gravity
measure, for instance with a Gaussian impedance function or an inverse
cost impedance function .

We use a simple numerical example to introduce the key concepts, and we
will use the usual accessibility measure for comparison. In this way, we
aim to show the differences between accessibility and spatial
availability, which helps to explain how spatial availability can
improve interpretability in the analysis of spatially dispersed
opportunities.

\hypertarget{numerical-example}{%
\subsection{Numerical Example}\label{numerical-example}}

In this section we present a simple numerical example. The setup for the
example is a system with three employment centers and nine population
centers, as seen in Table \ref{tab:toy-example}.

\begin{table}

\caption{\label{tab:toy-example-table}\label{tab:toy-example}Numerical example}
\centering
\begin{tabular}[t]{lrl>{}l}
\toprule
id & number & type & \\
\midrule
E1 & 750 & jobs & \\

E2 & 2250 & jobs & \\

E3 & 1500 & jobs & \\

P1 & 260 & population & \\

P2 & 255 & population & \\

P3 & 510 & population & \\

P4 & 495 & population & \\

P5 & 1020 & population & \\

P6 & 490 & population & \\

P7 & 980 & population & \\

P8 & 260 & population & \\

P9 & 255 & population & \multirow{-12}{*}{\raggedright\arraybackslash \includegraphics{images/figure-1.png}}\\
\bottomrule
\end{tabular}
\end{table}

The accessibility to employment of each of the population centers can be
calculated using the expression above for \(A_i\). As noted, this yields
the number of jobs (opportunities) that are accessible (i.e., can be
reached) from each population center, given the cost. In this example we
use the straight line distance between the population and jobs for
\(c_{ij}\), and a negative exponential function with \(\beta = 0.0015\).

\begin{figure}
\includegraphics[width=1\linewidth]{Spatial-Availability_files/figure-latex/toy-example-accessibility-plot-1} \caption{\label{fig:toy-example-accessibility}Accessibility results using simple numerical example}\label{fig:toy-example-accessibility-plot}
\end{figure}

Figure \ref{fig:toy-example-accessibility} shows the three employment
centers locations (black circles), where the size of the symbol is in
proportion to the number of jobs at each location. We also see nine
population centers (triangles), where the size of the symbol is
proportional to the accessibility (\[A_i\]) to jobs. At a glance:

\begin{itemize}
\item
  Population centers (triangles) in the middle of the map are relatively
  close to all three employment centers and thus have the highest levels
  of job accessibility. Population center P5 is relatively central and
  close to all employment centers, and it is the closest population to
  the second largest employment center in the region. Unsurprisingly,
  this population center has the highest accessibility 680.64);
\item
  Population centers (triangles) near the left edge of the map (only in
  proximity to the small employment center) have the lowest levels of
  job accessibility. Population center P1 is quite peripheral and the
  closest employment center is also the smallest one. Consequently, it
  has the lowest accessibility with \(A_i=\) 17.12);
\end{itemize}

\hypertarget{what-are-the-issues}{%
\subsection{What are the Issues?}\label{what-are-the-issues}}

Accessibility measures are excellent indicators of the intersection
between urban structure and transportation infrastructure. However,
beyond enabling comparisons of relative values they are not highly
interpretable on their own. For instance, from Figure
\ref{fig:toy-example-accessibility}, P1 has lower accessibility than P5,
however, despite the accessibility value for P1 being low it is still
better than \emph{zero}. This evaluation leaves decision makers unclear
on how to interpret accessibility values, particularly extreme values
and particularly when comparing across scenarios or regions . The
obscurity in interpretation of accessibility values arises from the
following two ways.

First, origins with accessibility to a high density of opportunities are
susceptible to having disproportionately higher accessibility values
than origins with a lower density of opportunities. This occurs since
total accessibility (\(\sum_{i=1}^IA_i\)) depends on the number of
origins; every additional origin in the analysis causes at least one and
possibly more opportunities at destinations to be multiple-counted. As
opportunities are multiple-counted, origins which are in proximity to
high density of opportunities have accessibility values which are
relatively conglomerated compared to origins in proximity to a lower
density of opportunities. This issue means total accessibility is
vulnerable to the modifiable areal unit problem (MAUP) as opportunities
and origins which are spatially aggregated are multiple-counted
inconsistently.

Second, the calculated accessibilities do not take competition into
account. For example, an individual at P5 has accessibility to
680.6373657 jobs. But since this is also a large population center,
there is potentially large competition for those accessible jobs. In
other words, the value of \(A_i\) is not sensitive to the size of
population at the origin seeking the opportunity (in this case jobs),
let alone the population at other locations. This unfortunately limits
the interpretability of the measure. Floating catchment areas purport to
account for competition/congestion effects, but as discussed by Paez,
Higgins, and Vivona (2019), they are vulnerable to conglomeration, which
makes them prone to bias unless corrected.

To address these two shortcomings of the accessibility measure, we
propose a singly-constrained gravity measure that corresponds to the
concept of \emph{spatial availability}.

\hypertarget{spatial-availability-proportional-allocation-of-opportunities-based-on-demand}{%
\section{Spatial Availability: Proportional Allocation of Opportunities
Based on
Demand}\label{spatial-availability-proportional-allocation-of-opportunities-based-on-demand}}

\hypertarget{analytical-framework}{%
\subsection{Analytical Framework}\label{analytical-framework}}

As recent research on accessibility shows (see for instance Paez,
Higgins, Vivona (2019) and Allen and Farber (2019)), accounting for
competitive access in a meaningful way requires the proportional
allocations of quantities in the accessibility calculations, or their
normalization. At issue is the fact that multiple-counting is
commonplace when calculating conventional accessibility \(A_i\) for
\(i=1,\cdots,n\) as every opportunity enters the weighted sum once for
every origin \(i\) that can reach it. This has the unfortunate effect of
obscuring the interpretability of \(A_i\) and fails to answer for a
individual at a specific population center the question: \emph{``many
jobs are accessible, but the same jobs are also accessible to my
(possibly) numerous neighbors\ldots what does high accessibility
actually mean to me?''}

In the spatial availability framework proposed, and in line with the
gravity tradition, we distinguish between opportunities at a destination
and demand for opportunities at the origin. To explain the analytical
framework, the example of access to employment is illustrative, with
``population'' in the role of demand (i.e., the number of individuals in
the labour market who ``demand'' employment) and ``jobs'' in the role of
opportunities.

As an overview, spatial availability (\(V_{ij}\)) is the proportional
allocation of opportunities (\(O\)) and allocation factor for population
(\(f^p\)) and cost of travel (\(f^c\)). Since spatial availability
(\(V_{ij}\)) consists of these two allocation factors, this example
first details how the population allocation factor \(f^p_{ij}\) produces
\(V^p_{ij}\), next details the role of travel cost allocation factor
\(f^c_{ij}\) in producing \(V^c_{ij}\), and finally combines both
allocation factors in the final general form of spatial availability
\(V_{ij}\) as follows:

\[
V_{ij} = O_j\frac{f^p_{ij} \cdot f^c_{ij}}{\sum_{k=1}^K f^p_{kj} \cdot f^c_{kj}}
\]

\hypertarget{population-allocation-factor}{%
\subsubsection{Population Allocation
Factor}\label{population-allocation-factor}}

We begin with allocation based on demand; consider an employment center
\(j\) with \(O_j^r\) jobs of type \(r\). In the general case where there
are \(K\) population centers in the region, the following factor can be
defined:

\[
f^p_{ij} = \frac{P_{i\in r}^\alpha}{\sum_{k=1}^K P_{k\in r}^\alpha}
\]

\noindent On the right hand side of the equation, \(P_{i\in r}\) is the
population at location \(i\) that is eligible for jobs of type \(r\)
(maybe those with a certain level of training, or in a designated age
group). The summation in the bottom is over \(k=1,\cdots,K\), the number
of population centers in the region. The resulting factor \(f^p_{ij}\)
corresponds to the proportion of the population in zone \(i\) relative
to the rest of the region's population centres \(K\). The factors
\(f^p_{ij}\) satisfy the property that \(\sum_i^{I} f^p_{ij} = 1\). We
can also add an empirical parameter \(\alpha\) that can be used to
modulate the effect of size in the calculations (i.e., \(\alpha <1\)
places greater weight on smaller centres relative to larger ones while
\(\alpha>1\) achieves the opposite effect).

This factor (\(f^p_{ij}\)) can now be used to proportionally allocate a
share of the jobs at the employment centre \(j\) to population center
\(i\) and population center \(k\). The share of jobs at \(j\) allocated
to (i.e., available to) each population center is:

\[
V^p_{ij} = O_jf^p_{ij}\\
\]

\noindent and since \(\sum_i^{I} f^p_{ij} = 1\) it follows that:

\[
\sum_{i=1}^I V_{ij} = O_j 
\]

In other words, the number of jobs is preserved. The result is a
proportional allocation of available jobs to population centres based on
demand. As an example, consider an employment center \(j\) in a region
with two population centers (say \(i\) and \(k\)). For simplicity,
assume that the all the population in the region is eligible for these
jobs, that is, that the entirety of the population is included in the
set \(r\). The allocation factors for the jobs at \(j\) would be:

\[
\begin{array}{l}\
f^p_{ij} = \frac{P_i ^\alpha}{P_i^\alpha + P_k^\alpha}\\
f^p_{kj} = \frac{P_k^\alpha}{P_i^\alpha + P_k^\alpha}\\
\end{array}
\]

Suppose that there are three hundred jobs in the employment center
(\(W_j = 300\)), and that the populations are \(P_i=240\) and
\(P_k=120\). The jobs are allocated as follows (assuming that
\(\alpha=1\)):

\[
\begin{array}{l}\
V^p_{ij} = O_jf^p_{ij} = O_j\frac{P_i^\alpha}{P_i^\alpha + P_k^\alpha} = 300\frac{240}{240 + 120} = 300\frac{240}{360} = 200\\
V^p_{kj} = O_jf^p_{kj} = O_j\frac{P_k^\alpha}{P_i^\alpha + P_k^\alpha} = 300\frac{120}{240 + 120} = 300\frac{120}{360} = 100 \\
\end{array}
\]

It can be seen that proportionally more jobs are allocated to the bigger
center and also that the total number of jobs is preserved. However, the
factors above account for the total number of opportunities at the
destination (i.e., the number of jobs at the employment center), but
they do not account for their location relative to the population
centers. The proportional allocation procedure above is insensitive to
how far population centers \(i\) or \(k\) are from employment center
\(j\). To account for this effect we define a second set of allocation
factors based on distance to the employment centers.

\hypertarget{travel-cost-allocation-factor}{%
\subsubsection{Travel Cost Allocation
Factor}\label{travel-cost-allocation-factor}}

These are defined as:

\[
f^c_{ij} = \frac{f(c_{ij})}{\sum_{k=1}^K f(c_{ij})}\\
\]

\noindent where \(c_{ij}\) is the cost (e.g., the distance, travel time,
etc.) from population center \(i\) to employment center \(j\), and
\(f(\cdot)\) is an impedance function that is a monotonically decreasing
function of cost (\(c_{ij}\)); in other words, this allocation factor
(\(f^c_{ij}\)) serves to proportionally allocates more jobs to closer
locations through an impedance function. To illustrate, assume that the
impedance function is a negative exponential function as follows, and
assume that \(\beta\) (which modulates the steepness of the impedance
effect and is an empirical parameter) is the value of 1:

\[
f(c_{ij}) = \exp(-\beta c_{ij})\\
\]

Continuing the example, suppose that the distance from population center
\(i\) to employment center \(j\) is 0.6 km, and the distance from
population center \(k\) to employment center \(j\) is 0.3 km. Being
closer, we would expect more jobs to be allocated to the population of
\(k\). The jobs would be sorted as follows:

\[
\begin{array}{l}\
f^c_{ij} = \frac{\exp(-\beta D_{ij})}{\exp(-\beta D_{ij}) + \exp(-\beta D_{kj})}\\
f^c_{kj} = \frac{\exp(-\beta D_{kj})}{\exp(-\beta D_{ij}) + \exp(-\beta D_{kj})}\\
\end{array}
\]

This step normalizes the impedance between \(i\) and \(j\) and \(k\) and
\(j\) by the total impedance in the study area. Numerically, the jobs
allocation is:

\[
\begin{array}{l}\
V^c_{ij} = O_jf^c_{ij} = O_j\frac{\exp(-D_{ij})}{\exp(-D_{ij}) + \exp(-D_{kj})} = 300\frac{\exp(-0.6)}{\exp(-0.6) + \exp(-0.3)} = 3\times 0.426 = 127.8\\
V^c_{kj} = O_jf^c_{kj} =  O_j\frac{\exp(-D_{kj})}{\exp(-D_{ij}) + \exp(-D_{kj})} = 300\frac{\exp(-0.3)}{\exp(-0.6) + \exp(-0.3)} = 3\times  0.574 = 172.2\\
\end{array}
\]

A larger share of jobs (172.2 jobs) is allocated to the population
center (\(k\)) that is closest as assumed. As before, the sum of jobs
allocated to the population centers matches the total number of jobs
available. Nevertheless, in isolation, this step does not account for
the allocation of jobs based on demand.

\hypertarget{putting-spatial-availability-together}{%
\subsubsection{Putting Spatial Availability
Together}\label{putting-spatial-availability-together}}

We can combine the proportional allocation factors by population and
distance and calculated spatial availability (\(V_{ij}\)) as follows:

\[
V_{ij} = O_j\frac{f^p_{ij} \cdot f^c_{ij}}{\sum_{k=1}^K f^p_{kj} \cdot f^c_{kj}}
\]

When applied to the example of population center \(i\) and \(k\) (i.e.,
the demand) traveling to employment center \(j\) (i.e., the opportunity
\(O\)):

\[
\begin{array}{l}\
V_{ij} = O_j\cdot \frac{f^p_{ij} \cdot f^c_{ij}}{f^p_{ij} \cdot f^c_{ij} + f^p_{kj} \cdot f^c_{kj}} = 300 \frac{\big(\frac{2}{3} \big) \big(0.426 \big)}{\big(\frac{2}{3} \big) \big(0.426 \big) + \big(\frac{1}{3} \big) \big(0.574 \big)} = \big(300 \big)\big(\frac{0.284}{0.475} \big)= 179.4\\
V_{kj} = O_j\cdot \frac{f^p_{kj} \cdot f^c_{kj}}{f^p_{ij} \cdot f^c_{ij} + f^p_{ik} \cdot f^c_{ik}} = 300 \frac{\big(\frac{1}{3} \big) \big(0.574 \big)}{\big(\frac{2}{3} \big) \big(0.426 \big) + \big(\frac{1}{3} \big) \big(0.574 \big)}  = \big(300 \big)\big(\frac{0.191}{0.475} \big)= 120.6 \\
\end{array}
\]

Notice how fewer jobs are allocated to population center \(i\) compared
to the allocation by population only, to account for the higher cost of
reaching the employment center. On the other hand, distance alone
allocated more jobs to the closest population center (i.e., \(k\)), but
since it is smaller, it also gets a smaller share of the jobs overall.
Again, the sum of jobs at employment center \(j\) that are allocated to
population centers \(i\) and \(k\) simultaneously based on
\emph{population-} and \emph{distance-} based allocation is preserved
(i.e., \(W_{ij} + W_{kj} = W_j\)).

Availability is simply the sum of the above by origin:

\[
V_i = \sum_{j=1}^J V_{ij}
\]

This quantity represents opportunities (e.g., jobs) that can be reached
from \(i\) (i.e., they are accessible), and that are \emph{not}
allocated to a competitor: therefore the weighted sum of available
opportunities. Compare \(V_i\) to the singly-constrained gravity model
(see Wilson \href{https://doi.org/10.1068/a030001}{(1971)}). In essence,
\(V_i\) is the result of constraining \(A_i\) to match one of the
marginals in the origin-destination table, the known total of
opportunities.

Since the sum of opportunities is preserved in the procedures above, it
is possible to calculate a highly interpretable measure of spatial
availability per capita (call it lower-case \(v_i\)) as follows:

\[
v_i = \frac{V_i}{P_i}
\]

In the example above:

\[
\begin{array}{l}\
v_{ij} = \frac{V_{ij}}{P_i} =  \frac{179.4}{240}\\
v_{kj} =  \frac{V_{ik}}{P_k} =  \frac{120.6}{120}\\
\end{array}
\]

Less competition (\(P_k\) is the smallest population center in the
region) and being closer to the jobs clearly works in favor of
individuals at \(k\). Where the overall ratio of jobs to population in
the region is \(300/(240 + 120)=\) 0.83, the spatially available jobs
per capita at \(k\) is closer to unity.

In the following sections we use the same numerical example presented
above to illustrate how availability \(V_i\) is calculated. We conclude
by contrasting the two measures in the final section.

\hypertarget{st-use-case-available-jobs-for-the-working-population}{%
\subsection{1st Use Case: Available Jobs for The Working
Population}\label{st-use-case-available-jobs-for-the-working-population}}

In the following examples we use the same impedance function that was
used to illustrate the accessibility calculations in the numerical
example. The spatial availability calculations are implemented in the
function \texttt{sp\_avail}. The inputs are an Origin-Destination table
with labels for the origins (\texttt{o\_id}), labels for the
destinations (\texttt{d\_id}), the population (\texttt{pop)} and number
of opportunities (\texttt{opp}), an indicator for catchments or other
eligibility constraints (\texttt{r}), and a pre-calculated impedance
function (\texttt{f}). For this example, we assume that there are no
catchment restrictions by setting \texttt{r} to 1.

The value of the function (its output) is a vector with \(V_ij\) given
the inputs, that is, the opportunities available to \(i\) from \(j\):

\begin{verbatim}
# A tibble: 9 x 2
  Origin          V_i
  <fct>         <dbl>
1 Population 1   67.0
2 Population 2  414. 
3 Population 3  336. 
4 Population 4  745. 
5 Population 5 2081. 
6 Population 6  270. 
7 Population 7  256. 
8 Population 8  316. 
9 Population 9   14.9
\end{verbatim}

The following plot shows the estimates of spatial availability:

\begin{figure}
\includegraphics[width=1\linewidth]{Spatial-Availability_files/figure-latex/toy-example-availability-jobs-1} \caption{\label{fig:toy-example-availability-jobs}Spatial availability of jobs}\label{fig:toy-example-availability-jobs}
\end{figure}

We see that population center 5 has the highest level of spatial
availability, due to being a large population center that is moreover
relatively close to jobs. To improve the interpretability of this
measure, we first note that the regional measure of jobs per capita is
0.994. We then calculate the spatially available jobs per person at each
population center:

Plot the spatial availability per person:

\begin{figure}
\includegraphics[width=1\linewidth]{Spatial-Availability_files/figure-latex/toy-example-availability-jobs-per-capita-1} \caption{\label{fig:toy-example-availability-jobs-per-capita}Spatial availability of jobs per capita}\label{fig:toy-example-availability-jobs-per-capita}
\end{figure}

Some population centers have almost two jobs available per person
(compared to the overall regional value of approximately one job per
capita), while others have less than one job available per person. This
does not mean that people are not taking some of the jobs. It means that
controlling for the cost of reaching jobs, they are worse off than those
with more jobs spatially available.

\hypertarget{nd-use-case-available-workers-for-employment-centers}{%
\subsection{2nd Use Case: Available Workers for Employment
Centers}\label{nd-use-case-available-workers-for-employment-centers}}

We can also examine the pool of workers available to each employment
center by considering the workers as the opportunities and the jobs as
the population.

Calculate the spatial availability by proportionally allocating jobs to
workers (we refer to spatial availability in this case as \(W_ji\)) as
follows:

Plot the availability estimates:

\begin{figure}
\includegraphics[width=1\linewidth]{Spatial-Availability_files/figure-latex/toy-example-availability-workers-1} \caption{\label{fig:toy-example-availability-workers}Spatial availability of workers}\label{fig:toy-example-availability-workers}
\end{figure}

Plot the spatial availability of workers per job:

\begin{figure}
\includegraphics[width=1\linewidth]{Spatial-Availability_files/figure-latex/toy-example-availability-workers-per-job-1} \caption{\label{fig:toy-example-availability-workers-per-job}Spatial availability of workers per job}\label{fig:toy-example-availability-workers-per-job}
\end{figure}

\hypertarget{rd-use-case-available-jobs-for-specialized-working-populations}{%
\subsection{3rd Use Case: Available Jobs for Specialized Working
Populations}\label{rd-use-case-available-jobs-for-specialized-working-populations}}

In this section we introduce catchment/eligibility constraints. Due to
differences in educational achievement among the population, the jobs in
Employment Center 1 can only be taken by individuals in population
centers 1 and 2. Jobs in Employment Center 2 can be taken by individuals
in population centers 3, 4, 5, 7, and 8. Lastly, jobs in Employment
Center 3 require qualifications available only among individuals in
population centers 5, 6, 8, and 9.

Calculate the spatial availability by proportionally allocating
\emph{specialized} workers to jobs (we refer to spatial availability in
this case as \(V_ij_r\)):

\begin{figure}
\includegraphics[width=1\linewidth]{Spatial-Availability_files/figure-latex/toy-example-plot-catchments-1} \caption{\label{fig:toy-example-catchments}Catchments areas in numeric example}\label{fig:toy-example-plot-catchments}
\end{figure}

Plot the availability estimates:

\begin{figure}
\includegraphics[width=1\linewidth]{Spatial-Availability_files/figure-latex/toy-example-availability-with-catchments-1} \caption{\label{fig:toy-example-availability-with-catchments}Spatial availability of jobs with catchment restrictions}\label{fig:toy-example-availability-with-catchments}
\end{figure}

Available jobs per person with catchment/eligibility conditions:

The plot in Fig.
\ref{fig:toy-example-availability-with-catchments-per-capita} shows the
availability per person without and with catchment restrictions.

\begin{figure}
\includegraphics[width=1\linewidth]{Spatial-Availability_files/figure-latex/toy-example-availability-with-catchments-per-capita-1} \caption{\label{fig:toy-example-availability-with-catchments-per-capita}Spatial availability of jobs per capita with and without catchment restrictions }\label{fig:toy-example-availability-with-catchments-per-capita}
\end{figure}

We can see that when there are catchment restrictions population center
2, despite being relatively peripheral, has higher spatial availability
due to spatial specialization. With catchments, the spatial availability
of jobs declines from the perspective of population center 4: the
population here has skills required for jobs at a small employment
center, where they face substantial competition from other population
centers.

\hypertarget{empirical-example-spatial-availability-and-accessibility-of-jobs-in-toronto}{%
\section{Empirical Example: Spatial Availability and Accessibility of
Jobs in
Toronto}\label{empirical-example-spatial-availability-and-accessibility-of-jobs-in-toronto}}

For the reasons demonstrated in the hypothetical toy example; the
spatial availability measure produces different and more interpretable
results than accessibility. In this section we will demonstrate both
measures and a comparision using empirical data for home-based work
trips to places of employment in Toronto, Ontario.

\hypertarget{data}{%
\subsection{Data}\label{data}}

The 2016 Transportation Tomorrow Survey (TTS) data for 20 municipalities
contained within the the Greater Golden Horseshoe (GGH) area in the
province of Ontario, Canada (43.6°N 79.73°W) is analysed (Figure
\ref{fig:TTS-16-survey-area}). This data set includes home origins and
work destinations defined by centroids of Traffic Analysis Zones (TAZ)
of varying areas (n=3764), the number of jobs (n=3081900) and workers
(3446957) at each origin and destination, and the trips from origin to
destination for the morning home-to-work commute (n=3069541). Also
included are calculated travel times by car (calculated via
\href{https://github.com/ipeaGIT/r5r}{\texttt{r5r}}) and a derived
impedance function values corresponding to the cost of travel based on
the trip length distribution (TLD). The descriptive statistics are
presented in Table \ref{tab:TTS-16-desc-stats} and a
\href{https://github.com/soukhova/AccessPack}{data-package} is available
to explore the data in greater detail.

\begin{figure}
\includegraphics[width=1\linewidth]{images/Greater-Golden-Horseshoe-Map} \caption{\label{fig:TTS-16-survey-area}The TTS 2016 study area within the Greater Golden Horseshoe in Ontario, Canada.}\label{fig:TTS-16-survey-area}
\end{figure}

\begin{table}

\caption{\label{tab:unnamed-chunk-1}\label{tab:TTS-16-desc-stats}Descriptive statistics of the TTS 2016 dataset for the Greater Golden Horshoe Area}
\centering
\begin{tabular}[t]{l|l|l}
\hline
  & Trips & Travel\_Time\\
\hline
 & Min.   :   1.00 & Min.   :  0.10\\
\hline
 & 1st Qu.:  14.00 & 1st Qu.: 13.00\\
\hline
 & Median :  22.00 & Median : 20.00\\
\hline
 & Mean   :  30.83 & Mean   : 23.39\\
\hline
 & 3rd Qu.:  37.00 & 3rd Qu.: 30.00\\
\hline
 & Max.   :1129.00 & Max.   :179.00\\
\hline
\end{tabular}
\end{table}

\hypertarget{calibrating-an-impedance-function}{%
\subsection{Calibrating an Impedance
Function}\label{calibrating-an-impedance-function}}

In the hypothetical toy data set, an arbitrary negative exponential
function describing the increase in travel cost as distance increases
was used as the impedance function to derive both accessibility and
availability. In this data set, an impedance function can be derived
from the empirical trip length distribution (TLD) as the number of trips
and their travel cost (in the case, travel time in minutes) are known
(see black points in Figure \ref{fig:TLD-Gamma-plot}).

The TLD density plot appears to follow a gamma distribution, as such,
this theoretical distribution along with other common TLD distributions
such as log-normal and exponential distributions were fitted to the
empirical TLD using the maximum likelihood estimation method and
Nelder-Mead method for direct optimization available within the
{[}`fitdistrplus' package{]}
(https://cran.r-project.org/web/packages/fitdistrplus/fitdistrplus.pdf)
in R . Based on goodness-of-fit criteria, the gamma distribution was
selected (see red line in Figure \ref{fig:TLD-Gamma-plot}) with the
following shape and rate parameters (Table \ref{tab:gamma-param}) in the
general form of: \[ 
f(x, \alpha, \beta) = \frac {x^{\alpha-1}e^{-\frac{x}{\beta}}}{ \beta^{\alpha}\Gamma(\alpha)} \quad \text{for } 0 \leq x \leq \infty
\] \noindent where the estimated `shape' is \(\alpha\),the estimated
`rate' is \(\beta\), and the \(\Gamma(\alpha)\) is defined as:

\[
\Gamma(\alpha) =  \int_{0}^{\infty} x^{\alpha-1}e^{-x} \,dx
\]

\begin{figure}
\includegraphics[width=1\linewidth]{images/impedance_function} \caption{\label{fig:impedance-function-plot}Impedance function and diagnostics.}\label{fig:plot-impedance-function}
\end{figure}

The calibrated impedance function has a shape parameter of 2.019 and a
rate parameter of 0.094. The function and diagnostics are plotted in
Fig. \ref{fig:impedance-function-plot}.

\hypertarget{discussion}{%
\section{Discussion}\label{discussion}}

\hypertarget{comparing-accessibility-and-spatial-availability}{%
\subsection{Comparing Accessibility and Spatial
Availability}\label{comparing-accessibility-and-spatial-availability}}

Side by side plots of accessibility and SA which are indexed by 0-100.
Use text below for inspiration.

Firstly, Accessibility - not interpetable, (XX accessible jobs per
person.. so what?) vs.~Availability (XX available jobs per person). We
can use the boundary benchmark to determine if an allocation is unjust
if it is below XX we couldn't do it with Accessibility because we can
compare by capita.

Secondly, scales back competition.

\hypertarget{discussion-1}{%
\section{Discussion}\label{discussion-1}}

\hypertarget{comparing-accessibility-and-spatial-availability-1}{%
\subsection{Comparing Accessibility and Spatial
Availability}\label{comparing-accessibility-and-spatial-availability-1}}

Below we compare accessibility and the first case of spatial
availability illustrated above. To facilitate the comparison, we index
the measures so that the lowest value corresponds to an index of 100:

The figure suggests that high accessibility does not necessarily mean
high availability.

\hypertarget{how-are-the-these-measures-different}{%
\subsection{How are the these measures
different?}\label{how-are-the-these-measures-different}}

The sum of all spatial availability values is equal to the total number
of opportunities within the region. As such, the value for each
population center reflects how many opportunities are \emph{available}
accounting for the indivisibility of the opportunities. This provides
greater interpretability of the results. Comparatively, accessibility
values associated with each population centers reflects how many jobs
can \emph{potentially} be reached by each population center; these
values are not adjusted proportionally to the number of population
(i.e.~workers ).

Secondly, spatial availability is not robust to the modifiable unit
problem, since the estimates account for the population at the centers.
If more population centers are added, the availability adjusts
accordingly by allocating opportunities proportionally.

Further, the measure of spatial availability can be a useful way to
distinguish between high accessibility/high population centers (which
potentially can result in lower availability due to competition), and
contrariwise, low accessibility/low population centers, which may enjoy
higher availability than the accessibility calculations may suggest.
Remote, smaller population centers can be sufficiently accessible and in
close proximity to the smaller employment centers; however this
``sufficiency'' is obscured by accessibility measure by over-inflating
the accessibly of population centers which are more central to more (and
larger) employment centers. Conventional accessible does not shed light
on how sufficiently accessible opportunities are available to the
population.

As a final point, we note that the measure proposed, by producing a
concrete and interpretable number of opportunities available, can be
meaningfully compared to the total number of opportunities in the
region. Likewise for opportunities available per capita. This is an
important topic in equity analysis. For instance, considering the two
remote small population centers in the top left corner it is evident
that for individuals in population center 2, despite facing low nominal
accessibility, have a relatively high number of available jobs per
capita.

It is a form of \emph{spatial mismatch} and related to \emph{balanced
floating catchment?} approach.

\hypertarget{use-cases-and-limitations-for-spatial-availability}{%
\subsection{Use Cases and Limitations for Spatial
Availability}\label{use-cases-and-limitations-for-spatial-availability}}

\hypertarget{conclusion}{%
\section{Conclusion}\label{conclusion}}

Words go here.

\hypertarget{references}{%
\section*{References}\label{references}}
\addcontentsline{toc}{section}{References}

\hypertarget{refs}{}
\begin{CSLReferences}{1}{0}
\leavevmode\vadjust pre{\hypertarget{ref-allen2019}{}}%
Allen, J., Farber, S., 2019. A Measure of Competitive Access to
Destinations for Comparing Across Multiple Study Regions. Geographical
Analysis 52, 69--86.
doi:\href{https://doi.org/10.1111/gean.12188}{10.1111/gean.12188}

\leavevmode\vadjust pre{\hypertarget{ref-deboosere2018}{}}%
Deboosere, R., El-Geneidy, A.M., Levinson, D., 2018.
Accessibility-oriented development. Journal of Transport Geography 70,
11--20.
doi:\href{https://doi.org/10.1016/j.jtrangeo.2018.05.015}{10.1016/j.jtrangeo.2018.05.015}

\leavevmode\vadjust pre{\hypertarget{ref-delamater2013spatial}{}}%
Delamater, P.L., 2013. Spatial accessibility in suboptimally configured
health care systems: A modified two-step floating catchment area
(M2SFCA) metric. Health \& Place 24, 30--43.
doi:\href{https://doi.org/10.1016/j.healthplace.2013.07.012}{10.1016/j.healthplace.2013.07.012}

\leavevmode\vadjust pre{\hypertarget{ref-geurs2004}{}}%
Geurs, K.T., van Wee, B., 2004. Accessibility evaluation of land-use and
transport strategies: review and research directions. Journal of
Transport Geography 12, 127--140.
doi:\href{https://doi.org/10.1016/j.jtrangeo.2003.10.005}{10.1016/j.jtrangeo.2003.10.005}

\leavevmode\vadjust pre{\hypertarget{ref-handy2020}{}}%
Handy, S., 2020. Is accessibility an idea whose time has finally come?
Transportation Research Part D: Transport and Environment 83, 102319.
doi:\href{https://doi.org/10.1016/j.trd.2020.102319}{10.1016/j.trd.2020.102319}

\leavevmode\vadjust pre{\hypertarget{ref-hansen1959}{}}%
Hansen, W.G., 1959. How Accessibility Shapes Land Use. Journal of the
American Institute of Planners 25, 73--76.
doi:\href{https://doi.org/10.1080/01944365908978307}{10.1080/01944365908978307}

\leavevmode\vadjust pre{\hypertarget{ref-joseph1984}{}}%
Joseph, A.E., Bantock, P.R., 1984. Rural Accessibility of General
Practitioners: the Case of Bruce and Grey Counties, ONTARIO,
1901{\textendash}1981. The Canadian Geographer/Le Géographe canadien 28,
226--239.
doi:\href{https://doi.org/10.1111/j.1541-0064.1984.tb00788.x}{10.1111/j.1541-0064.1984.tb00788.x}

\leavevmode\vadjust pre{\hypertarget{ref-luo2003}{}}%
Luo, W., Wang, F., 2003. Measures of Spatial Accessibility to Health
Care in a GIS Environment: Synthesis and a Case Study in the Chicago
Region. Environment and Planning B: Planning and Design 30, 865--884.
doi:\href{https://doi.org/10.1068/b29120}{10.1068/b29120}

\leavevmode\vadjust pre{\hypertarget{ref-miller2018}{}}%
Miller, E.J., 2018. Accessibility: measurement and application in
transportation planning. Transport Reviews 38, 551--555.
doi:\href{https://doi.org/10.1080/01441647.2018.1492778}{10.1080/01441647.2018.1492778}

\leavevmode\vadjust pre{\hypertarget{ref-paez2019}{}}%
Paez, A., Higgins, C.D., Vivona, S.F., 2019. Demand and level of service
inflation in Floating Catchment Area (FCA) methods. PLOS ONE 14,
e0218773.
doi:\href{https://doi.org/10.1371/journal.pone.0218773}{10.1371/journal.pone.0218773}

\leavevmode\vadjust pre{\hypertarget{ref-proffitt2017}{}}%
Proffitt, D.G., Bartholomew, K., Ewing, R., Miller, H.J., 2017.
Accessibility planning in American metropolitan areas: Are we there yet?
Urban Studies 56, 167--192.
doi:\href{https://doi.org/10.1177/0042098017710122}{10.1177/0042098017710122}

\leavevmode\vadjust pre{\hypertarget{ref-shen1998}{}}%
Shen, Q., 1998. Location characteristics of inner-city neighborhoods and
employment accessibility of low-wage workers. Environment and Planning
B: Planning and Design 25, 345--365.
doi:\href{https://doi.org/10.1068/b250345}{10.1068/b250345}

\leavevmode\vadjust pre{\hypertarget{ref-wan2012three}{}}%
Wan, N., Zou, B., Sternberg, T., 2012. A three-step floating catchment
area method for analyzing spatial access to health services.
International Journal of Geographical Information Science 26,
1073--1089.
doi:\href{https://doi.org/10.1080/13658816.2011.624987}{10.1080/13658816.2011.624987}

\leavevmode\vadjust pre{\hypertarget{ref-wilson1971}{}}%
Wilson, A.G., 1971. A Family of Spatial Interaction Models, and
Associated Developments. Environment and Planning A: Economy and Space
3, 1--32. doi:\href{https://doi.org/10.1068/a030001}{10.1068/a030001}

\leavevmode\vadjust pre{\hypertarget{ref-yan2021}{}}%
Yan, X., 2021. Toward Accessibility-Based Planning. Journal of the
American Planning Association 87, 409--423.
doi:\href{https://doi.org/10.1080/01944363.2020.1850321}{10.1080/01944363.2020.1850321}

\end{CSLReferences}


\end{document}
