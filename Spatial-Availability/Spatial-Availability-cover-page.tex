\documentclass[]{elsarticle} %review=doublespace preprint=single 5p=2 column
%%% Begin My package additions %%%%%%%%%%%%%%%%%%%
\usepackage[hyphens]{url}

  \journal{Journal of Transport Geography} % Sets Journal name


\usepackage{lineno} % add

\usepackage{graphicx}
%%%%%%%%%%%%%%%% end my additions to header

\usepackage[T1]{fontenc}
\usepackage{lmodern}
\usepackage{amssymb,amsmath}
\usepackage{ifxetex,ifluatex}
\usepackage{fixltx2e} % provides \textsubscript
% use upquote if available, for straight quotes in verbatim environments
\IfFileExists{upquote.sty}{\usepackage{upquote}}{}
\ifnum 0\ifxetex 1\fi\ifluatex 1\fi=0 % if pdftex
  \usepackage[utf8]{inputenc}
\else % if luatex or xelatex
  \usepackage{fontspec}
  \ifxetex
    \usepackage{xltxtra,xunicode}
  \fi
  \defaultfontfeatures{Mapping=tex-text,Scale=MatchLowercase}
  \newcommand{\euro}{€}
\fi
% use microtype if available
\IfFileExists{microtype.sty}{\usepackage{microtype}}{}
\bibliographystyle{elsarticle-harv}
\ifxetex
  \usepackage[setpagesize=false, % page size defined by xetex
              unicode=false, % unicode breaks when used with xetex
              xetex]{hyperref}
\else
  \usepackage[unicode=true]{hyperref}
\fi
\hypersetup{breaklinks=true,
            bookmarks=true,
            pdfauthor={},
            pdftitle={Introducing spatial availability, a singly-constrained competitive-access accessibility measure},
            colorlinks=false,
            urlcolor=blue,
            linkcolor=magenta,
            pdfborder={0 0 0}}
\urlstyle{same}  % don't use monospace font for urls

\setcounter{secnumdepth}{0}
% Pandoc toggle for numbering sections (defaults to be off)
\setcounter{secnumdepth}{0}


% tightlist command for lists without linebreak
\providecommand{\tightlist}{%
  \setlength{\itemsep}{0pt}\setlength{\parskip}{0pt}}






\begin{document}


\begin{frontmatter}

  \title{Introducing spatial availability, a singly-constrained
competitive-access accessibility measure}
    \author[SEES]{Anastasia Soukhov}
   \ead{soukhoa@mcmaster.ca} 
    \author[SEES]{Antonio Paez\corref{Corresponding Author}}
   \ead{paezha@mcmaster.ca} 
    \author[UofTS]{Christopher D. Higgins}
   \ead{cd.higgins@utoronto.ca} 
    \author[CIVENG]{Moataz Mohamed}
   \ead{mmohame@mcmaster.ca} 
      \address[SEES]{School of Earth, Environment and Society, McMaster
University, Hamilton, ON, L8S 4K1, Canada}
    \address[University of Toronto Scarborough]{Department of Geography
\& Planning, University of Toronto Scarborough, 1265 Military Trail,
Toronto, ON M1C1A4}
    \address[CIVENG]{Dept. of Civil Engineering, McMaster University,
Hamilton, ON, L8S 4K1, Canada}
      \cortext[1]{Corresponding Author}
  
  \begin{abstract}
  Accessibility measures are widely used in transportation, urban, and
  health care planning, among other applications. These measures are
  weighted sums of the opportunities that can be reached given the cost
  of movement and are interpreted as the potential for spatial
  interaction. These measures are useful to understand spatial structure
  but double counting of opportunities leads to interpretability issues,
  as noted in recent research on balanced floating catchment areas
  (BFCA) and competitive measures of accessibility. In this paper we
  propose a new measure of \emph{spatial availability} which is
  calculated by imposing a single constraint on conventional
  gravity-based accessibility. Similar to the gravity model from which
  it is derived, a single constraint ensures that the marginals at the
  destination are met and thus the number of opportunities are
  preserved. Through examples, we detail the formulation of the proposed
  measure. Further, we use data from the 2016 travel survey in the
  Greater Toronto and Hamilton Area in Canada to contrast how
  conventional accessibility overestimates and underestimates the number
  of jobs \emph{available} to workers. We conclude with some discussion
  of the possible uses of spatial availability. Overall, we argue that
  spatial availability can be a more meaningful and interpretable
  measure of opportunity access in relation to conventional
  accessibility. All data and code used in this research are openly
  available.
  \end{abstract}
  
 \end{frontmatter}




\end{document}
