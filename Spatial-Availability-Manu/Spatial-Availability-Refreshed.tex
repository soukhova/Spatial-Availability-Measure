\documentclass[]{elsarticle} %review=doublespace preprint=single 5p=2 column
%%% Begin My package additions %%%%%%%%%%%%%%%%%%%

\usepackage[hyphens]{url}

  \journal{Journal of Transport Geography} % Sets Journal name

\usepackage{lineno} % add

\usepackage{graphicx}
%%%%%%%%%%%%%%%% end my additions to header

\usepackage[T1]{fontenc}
\usepackage{lmodern}
\usepackage{amssymb,amsmath}
\usepackage{ifxetex,ifluatex}
\usepackage{fixltx2e} % provides \textsubscript
% use upquote if available, for straight quotes in verbatim environments
\IfFileExists{upquote.sty}{\usepackage{upquote}}{}
\ifnum 0\ifxetex 1\fi\ifluatex 1\fi=0 % if pdftex
  \usepackage[utf8]{inputenc}
\else % if luatex or xelatex
  \usepackage{fontspec}
  \ifxetex
    \usepackage{xltxtra,xunicode}
  \fi
  \defaultfontfeatures{Mapping=tex-text,Scale=MatchLowercase}
  \newcommand{\euro}{€}
\fi
% use microtype if available
\IfFileExists{microtype.sty}{\usepackage{microtype}}{}
\usepackage[]{natbib}
\bibliographystyle{plainnat}

\ifxetex
  \usepackage[setpagesize=false, % page size defined by xetex
              unicode=false, % unicode breaks when used with xetex
              xetex]{hyperref}
\else
  \usepackage[unicode=true]{hyperref}
\fi
\hypersetup{breaklinks=true,
            bookmarks=true,
            pdfauthor={},
            pdftitle={Introducing spatial availability, a singly-constrained competitive-access accessibility measure},
            colorlinks=false,
            urlcolor=blue,
            linkcolor=magenta,
            pdfborder={0 0 0}}

\setcounter{secnumdepth}{0}
% Pandoc toggle for numbering sections (defaults to be off)
\setcounter{secnumdepth}{0}


% tightlist command for lists without linebreak
\providecommand{\tightlist}{%
  \setlength{\itemsep}{0pt}\setlength{\parskip}{0pt}}



\usepackage[font=small,skip=0pt]{caption}
\usepackage{booktabs}
\usepackage{longtable}
\usepackage{array}
\usepackage{multirow}
\usepackage{wrapfig}
\usepackage{float}
\usepackage{colortbl}
\usepackage{pdflscape}
\usepackage{tabu}
\usepackage{threeparttable}
\usepackage{threeparttablex}
\usepackage[normalem]{ulem}
\usepackage{makecell}
\usepackage{xcolor}



\begin{document}


\begin{frontmatter}

  \title{Introducing spatial availability, a singly-constrained
competitive-access accessibility measure}
      \cortext[cor1]{Corresponding author}
  
  \begin{abstract}
  Accessibility measures are widely used in transportation, urban and
  healthcare planning, among other applications. These measures are
  weighted sums of the opportunities that can be reached given the cost
  of movement and are estimates of the potential for spatial
  interaction. These measures are useful in understanding spatial
  structure, but they do not properly account for competition due to
  multiple counting of opportunities. This leads to interpretability
  issues, as noted in recent research on balanced floating catchment
  areas (BFCA) and competitive measures of accessibility. In this paper,
  we respond to the limitations of accessibility analysis by proposing a
  new measure of \emph{spatial availability} which is calculated by
  imposing a single constraint on conventional gravity-based
  accessibility. Similar to the gravity model from which spatial
  availability is derived, a single constraint ensures that the
  marginals at the destination are met and thus the number of
  opportunities are preserved. Through examples, we detail the
  formulation of the proposed measure. Further, we use data from the
  2016 Transportation Tomorrow Survey of the Greater Golden Horseshoe
  area in southern Ontario, Canada, to contrast how the conventional
  accessibility measure tends to overestimate and underestimate the
  number of jobs \emph{available} to workers. We conclude with some
  discussion of the possible uses of spatial availability and argue
  that, compared to conventional measures of accessibility, it can offer
  a more interpretable measure of opportunity access. All data and code
  used in this research are openly available which should facilitate
  testing by other reserarchers in their case studies.
  \end{abstract}
  
 \end{frontmatter}

\newpage

\hypertarget{introduction}{%
\section{Introduction}\label{introduction}}

Accessibility analysis is employed in transportation, geography, public
health, and many other areas, particularly as mobility-based planning is
de-emphasized in favor of access-oriented planning
\citep{deboosere2018, handy2020, proffitt2017, yan2021}. The concept of
accessibility derives its appeal from combining the spatial distribution
of opportunities and the cost of reaching them \citep{hansen1959}.

Numerous methods for calculating accessibility have been proposed that
can be broadly organized into infrastructure-, place-, person-, and
utility-based measures \citep{geurs2004}. Of these, the place-based
family of measures is arguably the most common, capturing the number of
opportunities reachable from an origin using the transportation network.
A common type of accessibility measure is based on the gravity model of
spatial interaction; since it was first developed by \citet{hansen1959}
it has been widely adopted in many forms
\citep[e.g.,][]{cervero_transportation_2002, paez2004network, geurs2004, handy_measuring_1997, levinson_accessibility_1998, Arranz2019measuring}.
Accessibility analysis offers a powerful tool to study the intersection
between urban structure and transportation infrastructure - however, the
interpretability of accessibility measures can be challenging
\citep{geurs2004, miller2018}. A key issue is that accessibility
measures are sensitive to the number of opportunities in a region (e.g.,
a large city has more jobs than a smaller city), and therefore raw
values cannot be easily compared across study areas \citep{allen2019}.

Gravity-based accessibility indicators are in essence spatially smoothed
estimates of the total number of opportunities in a region, but the
meaning of their absolute magnitudes are unclear as they measure the
\emph{potential} for interaction (as originally defined by
\citeyearpar{wilson1971}XXX). For instance, in the study by (XXX)
high-income areas in Metropolitan Saul Paulo have high accessibility to
employment (e.g., X jobs per all jobs available) and low-income areas
have low accessibility to employment (e.g., Y jobs per all jobs
available). Though X or Y reflect the urban structure, and correlates
identify the relationship with employment, X and Y values can only be
used within the context of employment accessibility in Saul Paulo. For
example, in study XX which used the same employment accessibility method
but for ZZ region, the X and Y accessibility values from Saul Paul
cannot be immediately compared, but the X and Y low and high
accessibility values in ZZ reflect the region's urban structure. That
being said, decision-makers and planners find it difficult to
operationalize accessibility since it reflects urban structure and is
harder to interpret as an opportunity \emph{provision} measure. For
instance, if a small city creates 1000 more housing units in an
employment rich neighborhood (ex., 15\% of all employment opportunities
are located within the neighbourhood (X)), could the city achieve a
provider-provision-ratio of 1.5 jobs per person? Conventional
accessibility cannot be standardized or normalized to reflect this ratio
as a result of \emph{inconsistent} opportunity-side and population-side
counting.

This inconsistent counting embedded in traditional accessibility methods
result in lack of clairty on how to interpret the resulting values.
Returning to XX, the maximum job accessibility in Saul Paulo is XYZ, and
the maximum job accessibility is XY in region ZZ. This the highest count
of opportunities, weighted by associated travel costs, which a certain
population can accessing the region. This maximum value, outside of
reflecting urban structure, is difficult to interpret from a
decision-maker's perspective who is deliberating planning interventions.
We believe the issue in interpretation arises since opportunities are
adjusted based on population-side travel and usage behaviour (i.e.,
travel costs, catchments, time-windows). Accessibility is not
constrained to match the number of opportunities available. Put another
way, traditional measures of accessibility do not capture the
competition for opportunities but instead quantify access as if every
person can reach every (travel and usage behaviour-adjusted)
opportunity. This \emph{inconsistency} is not necessarily problematic if
a case study calls for the quantification of a accessibility assuming a
``greedy'' population. For instance, a greater accessibility to parks
correlates with property values even though those that live in those
properties don't use all the parks all the time, but this correlate
suggests that the \emph{potential} to access parks is enough. However,
this \emph{inconsistent} opportunity-adjustment can be acute when
opportunities are ``non-divisible'' in the sense that, once taken they
are no longer available to other members of the population \citep[also
discussed by][]{geurs2004}. Examples of these types of opportunities
include jobs (e.g., when a person takes up a job, the same job cannot be
taken by anyone else) and placements at schools (e.g., once a student
takes a seat at a school, that opportunity is no longer available for
another student). Though these non-divisible opportunities can still be
modelled by conventional accessibility (e.g., Saul Paul and Z region),
the measure does not reflect the \emph{spatial availability} of the
opportunity.

To re-calibrate accessibility for a more interpretable understanding of
opportunity-provision, researchers have proposed several different
approaches for considering competition in the conventional accessibility
method. These include several approaches that first normalize the number
of opportunities available at a destination by the demand from the
origin zones and, second, sum the demand-corrected opportunities which
can be reached from the origins \citep[e.g.][]{joseph1984, shen1998}.
These advances were popularized in the family of two-step floating
catchment area (FCA) methods \citep{luo2003} that have found widespread
adoption for calculating competitive accessibility to a variety of
opportunities such as healthcare, education, and food access
\citep{yang_comparing_2006, chen_spatial_2020, ye_spatial_2018, chen_enhancing_2019, chen_evaluating_2020}.
In principle, floating catchments purport to account for competition
effects, although in practice several researchers
\citep[e.g.,][]{delamater2013spatial, wan2012three} have found that they
tend to over-estimate accessibility values. The underlying issue in
FCAs, as demonstrated by \citet{paez2019}, is the multiple counting of
opportunities (ex., population 1 in a catchment accesses opportunities
and population 2 in an overlapping catchment access a few opportunities
which population 1 access as well as additional opportunities). The
multiple counting of the \emph{same} opportunities across populations
can lead to biased estimates if not corrected \citep{paez2019}.

Another approach which research have used to consider competition in
accessibility has been the imposition of constraints on the gravity
model to ensure potential interaction between zones are equal to the
observed totals. Based on Wilson's \citeyearpar{wilson1971}
entropy-derived gravity model, researchers can incorporate constraints
to ensure that the modeled flows match some known quantities in the data
inputs. In this way, models can be singly-constrained to match the row-
or column-marginals (i.e., the trips produced or attracted,
respectively), whereas a doubly-constrained model is designed to match
both marginals. Allen and Farber \citeyearpar{allen2019} recently
incorporated a version of the doubly-constrained gravity model within
the FCA approach to calculate competitive accessibility to employment
using transit across eight cities in Canada. But while such a model can
account for competition, the mutual dependence of the balancing factors
in a doubly-constrained model means they must be iteratively calculated
which makes them more computationally-intensive. Furthermore, the double
constraint means that the sum of opportunity-seekers and the sum of
opportunities must match, which is not necessarily true in every
potential use case (e.g., there might be more people searching for work
than jobs exist in a region).

In this paper we propose an alternative approach to measuring
competitive accessibility. We call it a measure of \textbf{spatial
availability}, and it aims to capture the number of opportunities that
are not only \emph{accessible} but also \emph{available} to the
opportunity-seeking population, in the sense that they have not been
claimed by a competing seeker of the opportunity. As we will show,
spatial availability is a singly-constrained measure of accessibility.
By allocating opportunities in a proportional way based on demand and
distance in a single step, this method avoids the issues that result
from multiple counting of opportunities in conventional accessibility
analysis. This method returns a measure of the rate of available
opportunities per opportunity-seeking population. Moreover, the method
also returns a benchmark value for the study region against which
results for individual origins can be compared both inter- and
intra-regionally and used in the context of opportunity provision
assessment. This novel approach comes at a time when the quantity and
resolution of data is exponentially increasing and the need to
operationalize accessibility methods in city-planning objectives is
urgent.

It should be noted that this is a novel approach which has yet to be
sufficiently refined for estimating purposes, however, there is room for
an additional competitive measure as prevailing measure do \emph{not}
always reflect trends seen in empirical case studies. XXX. Similarly,
conventional accessibility (i.e., potential interaction), fail to
reflect empirical observations in certain contexts as well. XXX.
Measures are highly depending on specific contexts and environments,
thus there's conceptually room for an additional measure.

This paper is split into two main parts. The first part introduces a
synthetic example that is used to calculated spatial availability
alongside conventional unconstrained accessibility and popular
competitive accessibility measures (XXX). The aim is to demonstrate the
formulation of the proposed measure and what it represents relative to
other accessibility measures. In the second part, we calculate, compare,
and contrast the spatial availability, conventional accessibility, and
Shen's competitive accessibility (XX) values for 2016 employment data in
the city of Toronto, Canada (Transportation Tomorrow Survey (TTS)). The
motivation of this part is to demonstrate how accessibility and a
popular competitive accessibility measure, both suffer from a lack of
interpretability. Additionally, the competitive accessibility measure
(XX) and spatial availability reinforce the ways in which accessibility
double-counts opportunities thus cautioning practitioners from using it
for opportunities where populations' are not greedy (i.e., 1 person can
only access 1 opportunity at any given time; the presence of many
opportunities doesn't matter when there are many neighbours since their
utilization is normalized). Finally, we conclude by cautioning the
limits of conventional accessibility measure use in its \emph{concept}
and outlining the advantages of the spatial availability measure and the
breadth of potential uses from the perspective of opportunity-provision
planning.

In the spirit of openness of research in the spatial sciences
\citep{brunsdon2021opening, paez2021open} this paper has a companion
open data product \citep{arribas2021Open}, and all code will be
available for replicability and reproducibility purposes.

\hypertarget{background}{%
\section{Part 1: Access measures on an synthetic
example}\label{background}}

We begin this part by introducing a synthetic example, calculating
conventional accessibility, and demonstrating the interpretation issues
associated with this widely used measure. We then present the proposed
\emph{spatial availability} measure, calculate the spatial availability
values for the synthetic example and discuss how the interpretation of
the resulting values from the perspective of opportunity-provision.
Spatial availability values are then compared and contrasted with other
competitive accessibility measures
\citep{shen1998, luo2003, horner_exploring_2004, allen2019} and the
interpretability of spatial availability is further elaborated.

\hypertarget{conventional-accessibility}{%
\subsection{Conventional
accessibility}\label{conventional-accessibility}}

The important works of \citeyearpar{harris_market_1954} and
\citeyearpar{hansen1959} provide a foundation for modeling
accessibility. From these seminal efforts, many accessibility measures
(excluding utility-based measures) have thus been derived from
\citeyearpar{wilson1971} influential work on the gravity model which
follows the formulation shown in Equation
(\ref{eq:conventional-accessibility}).

\begin{equation}
\label{eq:conventional-accessibility}
A_i = \sum_{j=1}^JO_j \cdot f(c_{ij})
\end{equation}

\noindent where:

\begin{itemize}
\tightlist
\item
  \(A\) is accessibility.
\item
  \(i\) is a set of origin locations.
\item
  \(j\) is a set of destination locations.
\item
  \(O_j\) is the number of opportunities at location \(j\);
  \(\sum_j O_j\) is the total supply of opportunities in the study
  region.
\item
  \(c_{ij}\) is a measure of the cost of moving between \(i\) and \(j\).
\item
  \(f(\cdot)\) is an impedance function of \(c_{ij}\); it can take the
  form of any monotonically decreasing function chosen based on positive
  or normative criteria \citep{paez2012measuring}.
\end{itemize}

As formally defined, accessibility \(A_i\) is the weighted sum of
opportunities that can be reached from location \(i\), given the cost of
travel \(c_{ij}\). Summing the opportunities in the neighborhood of
\(i\), as determined by the impedance function \(f(\cdot)\), provides
estimates of the number of opportunities that can be reached from \(i\)
at a certain cost. The type of accessibility can be modified depending
on the impedance function; for example, the measure could be cumulative
opportunities \citep[if \(f(\cdot)\) is a binary or indicator function
e.g.,][]{elgeneidy_cost_2016, rosik_forecast_2021, geurs2004, qi_decadelong_2018}
or a gravity measure using an impedance function modeled after any
monotonically decreasing function \citep[e.g., Gaussian, inverse power,
negative exponential, or log-normal, among others, see, \emph{inter
alia},][]{kwan_spacetime_1998, vale_influence_2017, reggiani_accessibility_2011, li_approach_2020}.
In practice, the accessibility measures derived from many cumulative and
gravity formulations tend to be highly correlated with one another
\citep{higgins2019, santanapalacios2022, kwan_spacetime_1998}.

The setup for our synthetic example is a system with two employment
centers and three population centers. Accessibility to jobs at each
population center is calculated using the accessibility measure \(A_i\)
in Equation (\ref{eq:conventional-accessibility}). In this synthetic
example, we use the straight line distance between the population and
jobs for \(c_{ij}\) and a negative exponential function with
\(\beta = 0.0015\). As noted, \(A_i\) represents the number of jobs
(i.e., opportunities) that can be reached from each population center
given the estimated cost as depicted in Figure
\ref{fig:toy-example-accessibility}.

\begin{figure}
\includegraphics[width=1\linewidth]{Spatial-Availability-Refreshed_files/figure-latex/toy-example-accessibility-plot-1} \caption{\label{fig:toy-example-accessibility}Accessibility to jobs (red text) from population centers (P) to employment centers (E) for the synthetic example. Values of population and employment are shown in white text.}\label{fig:toy-example-accessibility-plot}
\end{figure}

The conventional gravity accessibility measure has been shown to be an
excellent indicator of the intersection between urban structure and
transportation infrastructure
\citep{shi_literature_2020, reggiani_accessibility_2011, kwan_spacetime_1998}.
However, beyond enabling comparisons of relative values they are not
highly interpretable on their own \citep{miller2018}. For instance, from
Figure \ref{fig:toy-example-accessibility}, P1 has lower accessibility
than P2 but despite the accessibility value for P1 being relatively low
it is still better than \emph{zero}. On the other hand, P2 has high
accessibility, but is its accessibility excellent, good, or only fair?
What does it \emph{mean} for a location to have accessibility to so many
jobs?

To address this interpretability issue, previous research has aimed to
index and normalize values on a per demand-population basis
\citep[e.g.,][]{barboza_balancing_2021, pereira_distributional_2019, wang_access_2021}.
However, as recent research on accessibility discusses
\citep{allen2019, paez2019}, these steps do not address the bias
introduced through the uneven multiple-counting of opportunities and/or
population. The underlying issue arises as a result of the assumption
that for conventional accessibility \(A_i\), all opportunities are
\emph{available} to anyone from any origin \(i=1,\cdots,n\) who can
reach them: in other words, they are assumed to be infinitely divisible
and non-competitive. This results in every opportunity entering the
weighted sum once for every origin \(i\) that can reach it. Put another
way, if a densely populated population center pops up next to P2 this
center too will have a high accessibility score since \(A_i\) does not
consider competition of opportunities from neighbouring population
centers. Neglecting to constrain opportunity counts (i.e.,
single-constraint) in addition to obscuring the interpretability of
accessibility can also bias results spatially which will be discussed
later on in the paper.

\hypertarget{a-review-of-competitive-measures}{%
\subsection{A review of competitive
measures}\label{a-review-of-competitive-measures}}

Considering competition in accessibility is not new. The highly cited
work of \citeyearpar{shen1998} in which, in one step, he divides
conventional accessibility by the travel-cost adjusted population
seeking the opportunities in a given region. This work was in part
popularized by the 2-step floating catchment approach (2SFCA) introduced
by \citeyearpar{luo2003} and used today.

A synthetic example of a three population center and two employment
centre is solved using this popular competitive measure and shown in
Figure XX- left. The formulation of the 2SFCA approach is shown in step
1 (Equation (\ref{eq:2SFCA-step1})) where the PPR \(R_j\) is calculated
for each opportunity and then allocated to populations based on travel
cost \(f(\cdot)\) in step 2 (Equation (\ref{eq:2SFCA-step2})). The
synthetic example is solved in detail for the 2SFCA and all other
accessibility measures in the Appendix (XX).

\begin{equation}
\label{eq:2SFCA-step1}
R_{j} = \frac{O_{j}}{\sum_i P_{i} \cdot f(c_{ij})}\\
\end{equation}

\begin{equation}
\label{eq:2SFCA-step2}
A_{i} = {\sum_j R_{j} \cdot f(c_{ij})}\\
\end{equation}

\noindent where:

\begin{itemize}
\tightlist
\item
  \(A\) is accessibility.
\item
  \(i\) is a set of origin locations.
\item
  \(j\) is a set of destination locations.
\item
  \(O_j\) is the number of opportunities at location \(j\);
\item
  \(P_i\) is the population at location \(i\); \(\sum_j R_j\) is the
  total supply of opportunities in the study region.
\item
  \(R_j\) is the provider-to-population (PPR) ratio at location \(j\);
\item
  \(c_{ij}\) is a measure of the cost of moving between \(i\) and \(j\);
\item
  \(f(\cdot)\) is an impedance function of \(c_{ij}\).
\end{itemize}

Another type of competitive measure which has been used in the
literature include inverse balancing factors as used in
\citep{horner_exploring_2004} and \citep{allen2019}. This method
involves computing the inverse balancing factor for each origin through
an iterative procedure until it converges. The iterations seek to match
the travel-cost weighted opportunities to the travel-cost weighted
population in the region. It requires that the total population and
opportunities are equal in the region but the mean accessibilities from
previous iterations can be included to standardize the disbalance. The
solved synthetic example is in Figure XX - middle, and the formulation
of this method is as follows in Equation (\ref{eq:Inverse-balancing-1})
and (\ref{eq:Inverse-balancing-2}).

\begin{equation}
\label{eq:Inverse-balancing-1}
A_{i} = \frac{\bar A^{o}}{\bar A^{c}}{\sum_{j=1}^{J} \frac{O_{j}f(c_{ij})}{B_{j}}}\\
\end{equation}

\begin{equation}
\label{eq:Inverse-balancing-2}
B_{i} = {\sum_{i=1}^{I} \frac{P_{i}f(c_{ij})}{A_{i}}}\\
\end{equation}

\noindent where: - \(B_{i}\) is the balancing factor; other variables
defined in the gravity model.

Incorporating the concept of balance between the population and
opportunities, a recent advancement to the 2SFCA is the balanced 2 step
floating catchment approach (B2SFCA) by \citep{paez2019} is introduced.
In both steps, the number of opportunities from the PPR to each origin,
this results in a consistent number of opportunities being assigned but,
ultimately, the total number of opportunities is not preserved.

In the B2SFCA, the PPR \(R_{j}\) for each employment center can be
interpreted as the total number of jobs accessible to the total
population after being \emph{proportionally} adjusted to the travel
cost. The PPR \(R_{j}\) is then allocated, proportionally based on
travel cost, to each employment center. For this reason, the sum of all
\(A_{i}\) adds up to the same value as the sum of all \(R_{j}\). Since
PPR and the subsequent \(A_{i}\) are proportionally allocated based
travel costs, it should be noted that \(A_{i}\) no longer considers
\emph{potential} interaction in the how it was defined in the gravity
model \citep{hansen1959} and instead represent the allocation of PPR,
based on travel time, to each population. This measure introduces some
consistency in how the PPR is calculated (compared to the 2SFCA), but is
still lacking interpretability in the resulting values.

The solved synthetic example is in Figure XX - right, and the
formulation of this method is as follows in Equation (\ref{eq:B2SFCA-1})
and (\ref{eq:B2SFCA-2}).

\begin{equation}
\label{eq:B2SFCA-1}
R_{j} = \frac{O_{j}}{\sum_i P_{i} \frac{f(c_{ij})}{\sum_j f(c_{ij})}}\\
\end{equation}

\begin{equation}
\label{eq:B2SFCA-1}
A_{i} = {\sum_j R_{j}\frac{f(c_{ij})}{\sum_j f(c_{ij})}}\\
\end{equation}

\hypertarget{introducing-spatial-availability---model-formulation}{%
\subsection{Introducing spatial availability - model
formulation}\label{introducing-spatial-availability---model-formulation}}

Here we introduce the spatial availability model formulation and
demonstrate how it compares to competitive measures of accessibility.

Formally, spatial availability \(V_{i}\) is defined by the number of
opportunities \(O\) that are proportionally allocated based on a
population allocation factor \(F^p_{i}\) and cost of travel allocation
factor \(F^c_{ij}\) for all origins \(i\) to all destinations \(j\) as
detailed in Equation (\ref{eq:spatial-availability}). In line with the
tradition of gravity modeling, the proposed framework distinguishes
between opportunities at a destination and demand for opportunities at
the origin.

\begin{equation}
\label{eq:spatial-availability}
V_{i} = O_j\frac{F^p_{i} \cdot F^c_{ij}}{\sum_{i=1}^K F^p_{i} \cdot F^c_{ij}}
\end{equation}

The terms in Equation \ref{eq:spatial-availability} are as follows:

\begin{itemize}
\tightlist
\item
  \(V_{i}\) is the spatial availability of opportunities in \(j\) to
  origin \(i\).
\item
  \(i\) is a set of origin locations in the region \(K\).
\item
  \(j\) is a set of destination locations in the region \(K\).
\item
  \(O_j\) is the number of opportunities at location \(j\) in the region
  \(K\).
\item
  \(F^p_{i}\) is a proportional allocation factor of the population in
  \(i\).
\item
  \(F^c_{ij}\) is a proportional allocation factor of travel cost for
  \(i\); it is a product of a monotonically decreasing (i.e., impedance)
  function associated with the cost of travel between \(i\) and \(j\).
\end{itemize}

Notice that, unlike \(A_i\) in Equation
(\ref{eq:conventional-accessibility}), the population in the region
enters the calculation of \(V_{i}\). It is important to detail the role
of the two proportional allocations factors in the formulation of
spatial availability. We begin by considering the population allocation
factor \(F^p_{i}\) followed by the role of the travel cost allocation
factor \(F^c_{ij}\); then we show how both allocation factors combine in
the final general form of spatial availability \(V_{i}\). The
calculation of spatial availability is introduced with a step-by-step
example for synthetic three population centers (\(P_1\), \(P_2\),
\(P_4\)) in the role of demand (i.e., the number of individuals in the
labor market who `demand' employment) and two employment centers
(\(O_1\), \(O_2\)) in the role of opportunities.

\hypertarget{population-and-travel-cost-allocation-factors}{%
\subsection{Population and travel cost allocation
factors}\label{population-and-travel-cost-allocation-factors}}

We begin with allocation based on demand by population; consider an
employment center \(j\) with \(O_j^r\) jobs of type \(r\). In the
general case where there are \(K\) population centers in the region, we
define the following factor:

\begin{equation}
\label{eq:pop-alloc-factor}
F^p_{i} = \frac{P_{i\in r}^\alpha}{\sum_{i=1}^K P_{i\in r}^\alpha}
\end{equation}

The population allocation factor \(F^p_{i}\) corresponds to the
proportion of the population in origin \(i\) relative to the population
in the region. On the right hand side of the equation, the numerator
\(P_{i\in r}\) is the population at origin \(i\) that is eligible for
and `demands' jobs of type \(r\) (e.g., those with a certain level of
training or in a designated age group). The summation in the denominator
is over \(i=1,\cdots,K\), the population at origins \(i\) in the region.
To modulate the effect of demand by population in this factor we include
an empirical parameter \(\alpha\) (i.e., \(\alpha <1\) places greater
weight on smaller centers relative to larger ones while \(\alpha>1\)
achieves the opposite effect). This population allocation factor
\(F^p_{i}\) can now be used to proportionally allocate a share of the
jobs at \(j\) to origins.

More broadly, since the factor \(F^p_{i}\) is a proportion, when it is
summed over \(i=1,\cdots,K\) it always equals to 1 (i.e.,
\(\sum_i^{K} F^p_{i} = 1\)). This is notable since the share of jobs at
each destination \(j\) allocated to (i.e., available to) each origin,
based on population, is equal to \(V^p_{i} = O_j \cdot F^p_{i}\). Since
the sum of \(F^p_{i}\) is equal to 1, it follows that
\(\sum_{i=1}^I V_{i} = O_j\). In other words, the number of jobs across
the region is preserved. The result is a proportional allocation of jobs
(opportunities) to origins based on the size of their populations.

For simplicity, assume that all the population in the region is eligible
for these jobs, that is, that the entirety of the population is included
in the set \(r\). Also assume that \(\alpha=1\). The population
allocation factors \(F^p_{i}\) is as follows in Equation
(\ref{eq:pop-alloc-factor-2populations}).

\begin{equation}
\label{eq:pop-alloc-factor-2populations}
\begin{array}{l}
F^p_{1} = \frac{P_1 ^\alpha}{P_1^\alpha + P_2^\alpha + P_3^\alpha} = \frac{260}{260 + 255 + 495} = 0.257\\
F^p_{2} = \frac{P_2^\alpha}{P_1^\alpha + P_2^\alpha + P_3^\alpha}  = \frac{255}{260 + 255 + 495} = 0.252\\
F^p_{3} = \frac{P_3^\alpha}{P_1^\alpha + P_2^\alpha + P_3^\alpha}  = \frac{495}{260 + 255 + 495} = 0.490\\
\end{array}
\end{equation}

These \(F^p_{i}\) values can be used to find a \emph{partial} spatial
availability in which jobs are allocated proportionally to population;
this partial spatial availability \(V^p_{i}\) for each population center
is calculated as follows in Equation
(\ref{eq:pop-alloc-factor-SA-2populations}).

\begin{equation}
\label{eq:pop-alloc-factor-SA-2populations}
\begin{array}{l}
V^p_{1} = O_1 \cdot F^p_{1} + O_2 \cdot F^p_{1} = 750 \cdot 0.257 + 220 \cdot 0.257 = 249.29 \\
V^p_{2} = O_1 \cdot F^p_{2} + O_2 \cdot F^p_{2} = 750 \cdot 0.252 + 220 \cdot 0.252 = 244.44 \\
V^p_{3} = O_1 \cdot F^p_{3} + O_2 \cdot F^p_{3}= 750 \cdot 0.490 + 220 \cdot 0.490 = 475.30 \\
\end{array}
\end{equation}

When using only the proportional allocation factor \(F^p_{i}\) to
calculate spatial availability (differentiated here by being defined as
\(V^p_{i}\) instead of \(V_{i}\)), proportionally more jobs are
allocated to the bigger population center (i.e., 2 times more jobs as it
is 2 times larger in population). We can also see that the sum of
spatial availability for all population centers is equal to the sum of
jobs; put another way, the total opportunities are preserved.

Clearly, using only the proportional allocation factor \(F^p_{i}\) to
calculate spatial availability does not account for how far population
centers are from employment centers. It is the task of the second
allocation factor \(F^c_{ij}\) to account for the friction of distance,
as seen in Equation (\ref{eq:tcost-alloc-factor}).

\begin{equation}
\label{eq:tcost-alloc-factor}
F^c_{ij} = \frac{f(c_{ij})}{\sum_{i=1}^K f(c_{ij})}\\
\end{equation}

Travel cost allocation factor \(F^c_{ij}\) serves to proportionally
allocate more jobs to closer locations through an impedance function.
\(c_{ij}\) is the cost (e.g., the distance, travel time, etc.) to reach
employment center \(j\) from \(i\) and \(f(\cdot)\) is an impedance
function that depends on cost (\(c_{ij}\)).

To continue with the example, assume that the impedance function is a
exponential function with \(\beta=-0.00015\) and the distance from
population centers to employment centers is as shown in TABLE XX.
\(\beta\) modulates the steepness of the impedance effect and is
empirically determined in the case of positive accessibility, or set by
the analyst to meet a preset condition in the case of normative
accessibility \citep{paez2012measuring}. The proportional allocation
factor \(F^p_{i}\) for all population centers is defined in Equation
(\ref{eq:tcost-allocation-factor-2populations}).

\begin{equation}
\label{eq:tcost-allocation-factor-2populations}
\begin{array}{l}
F^c_{1,1} = \frac{\exp(\beta*2548.1)}{\exp(\beta *2548.1) + \exp(\beta *1314.1) + \exp(\beta *2170.2)} = 0.109\\
F^c_{2,1} = \frac{\exp(\beta *1314.1)}{\exp(\beta *2548.1) + \exp(\beta *1314.1) + \exp(\beta *2170.2)} = 0.697\\
F^c_{3,1} = \frac{\exp(\beta *2170.2)}{\exp(\beta *2548.1) + \exp(\beta *1314.1) + \exp(\beta *2170.2)} = 0.193\\
F^c_{1,2} = \frac{\exp(\beta*5419.1)}{\exp(\beta *5419.1) + \exp(\beta *2170.2) + \exp(\beta *1790.1)} = 0.004\\
F^c_{2,2} = \frac{\exp(\beta *4762.6)}{\exp(\beta *5419.1) + \exp(\beta *2170.2) + \exp(\beta *1790.1)} = 0.011\\
F^c_{3,2} = \frac{\exp(\beta *1790.1)}{\exp(\beta *5419.1) + \exp(\beta *2170.2) + \exp(\beta *1790.1)} = 0.984\\
\end{array}
\end{equation}

We can see, for instance, that the proportional allocation factor for
\(P_2\) is largest for \(E_1\) since the cost (i.e., distance) to
\(E_1\) is lowest. For \(E_2\), \(P_3\) has the largest proportional
allocation factor similarly because it is in the closest proximity.
Using the travel cost proportional allocation factors \(F^c_{ij}\) as
defined in Equation (\ref{eq:tcost-allocation-factor-2populations}), we
can calculate the spatial availability of jobs for each population
center based only on \(F^c_{ij}\) and the jobs available at each
employment center, as shown in Equation
(\ref{eq:tcost-allocation-factor-SA-2populations}).

\begin{equation}
\label{eq:tcost-allocation-factor-SA-2populations}
\begin{array}{l}
V^c_{1,1} = E_1 \cdot F^c_{1,1} = 750 \times 0.109 = 81.75\\
V^c_{2,1} = E_1 \cdot F^c_{2,1} = 750 \times  0.697 = 522.75\\
V^c_{3,1} = E_1 \cdot F^c_{3,1} = 750 \times  0.193 = 144.75\\
V^c_{1,2} = E_2 \cdot F^c_{1,2} = 220 \times 0.004 = 0.88\\
V^c_{2,2} = E_2 \cdot F^c_{2,2} = 220 \times  0.011 = 2.42\\
V^c_{3,2} = E_2 \cdot F^c_{3,2} = 220 \times  0.984 = 216.48\\
\end{array}
\end{equation}

\begin{verbatim}
[1] 81.75
\end{verbatim}

\begin{verbatim}
[1] 522.75
\end{verbatim}

\begin{verbatim}
[1] 144.75
\end{verbatim}

\begin{verbatim}
[1] 0.88
\end{verbatim}

\begin{verbatim}
[1] 2.42
\end{verbatim}

\begin{verbatim}
[1] 216.48
\end{verbatim}

For instance, spatial availability defined by \(F^c_{ij}\) only (i.e.,
\(V^c_{i}\)) allocates a largest share of jobs from \(E_1\) to \(P_2\)
since it is the closest. However, as previously discussed, \(P_2\) has a
relatively small population, so \(V^p_{2,1}\) is actually the smallest
value of any population center for \(E_1\). It is necessary to combine
both population and travel cost factors to better reflect demand; these
two components are in line with how demand is conventionally modelled in
accessibility calculations which are re-scaled on a per
demand-population basis or also consider competition
\citep[e.g.,][]{allen2019, barboza_balancing_2021, yang_comparing_2006}.
Fortunately, since both \(F^c_{ij}\) and \(F^p_{i}\) preserve the total
number of opportunities as they independently sum to 1, they can be
combined multiplicatively to calculate the proposed spatial availability
\(V_{i}\) which considers demand to be based on both population and
travel cost.

\hypertarget{putting-spatial-availability-together}{%
\subsection{Putting spatial availability
together}\label{putting-spatial-availability-together}}

We can combine the proportional allocation factors by population
\(F^p_{i}\) and travel cost \(F^c_{ij}\) and calculate spatial
availability \(V_{i}\) as introduced in Equation
(\ref{eq:spatial-availability}) and repeated below:

\[
V_{i} = O_j\frac{F^p_{i} \cdot F^c_{ij}}{\sum_{i=1}^K F^p_{i} \cdot F^c_{ij}}
\]

The resulting spatial availability \(V_{i}\) is calculated for all
population centers is calculated in Equation (\ref{eq:SA-2populations}).

\begin{equation}
\label{eq:SA-2populations}
\begin{array}{l}

V_{1,1} = O_1\cdot \frac{F^p_{1,1} \cdot F^c_{1,1}}{F^p_{1,1} \cdot F^c_{1,1} + F^p_{2,1} \cdot F^c_{2,1} + F^p_{3,1} \cdot F^c_{3,1}} = 
750 \cdot \frac{0.26 \cdot 0.109}{0.26 \cdot 0.109 + 0.25 \cdot 0.697 + 0.49 \cdot 0.193} = 70.45\\
V_{2,1} = O_1\cdot \frac{F^p_{2,1} \cdot F^c_{2,1}}{F^p_{1,1} \cdot F^c_{1,1} + F^p_{2,1} \cdot F^c_{2,1} + F^p_{3,1} \cdot F^c_{3,1}} = 
750 \cdot \frac{0.25 \cdot 0.697}{0.26 \cdot 0.109 + 0.25 \cdot 0.697 + 0.49 \cdot 0.193} = 441.72\\
V_{3,1} = O_1\cdot \frac{F^p_{3,1} \cdot F^c_{3,1}}{F^p_{1,1} \cdot F^c_{1,1} + F^p_{2,1} \cdot F^c_{2,1} + F^p_{3,1} \cdot F^c_{3,1}} = 
750 \cdot \frac{0.49 \cdot 0.193}{0.26 \cdot 0.109 + 0.25 \cdot 0.697 + 0.49 \cdot 0.193} = 237.83\\

V_{1,2} = O_2\cdot \frac{F^p_{1,2} \cdot F^c_{1,2}}{F^p_{1,2} \cdot F^c_{1,2} + F^p_{2,2} \cdot F^c_{2,2} + F^p_{3,2} \cdot F^c_{3,2}} = 
220 \cdot \frac{0.26 \cdot 0.004}{0.26 \cdot 0.004 + 0.25 \cdot 0.011 + 0.49 \cdot 0.984} = 0.46\\
V_{2,2} = O_2\cdot \frac{F^p_{2,2} \cdot F^c_{2,2}}{F^p_{1,2} \cdot F^c_{1,2} + F^p_{2,2} \cdot F^c_{2,2} + F^p_{3,2} \cdot F^c_{3,2}} = 
220 \cdot \frac{0.25 \cdot 0.011}{0.26 \cdot 0.004 + 0.25 \cdot 0.011 + 0.49 \cdot 0.984} = 1.26\\
V_{3,2} = O_2\cdot \frac{F^p_{1,2} \cdot F^c_{1,2}}{F^p_{1,2} \cdot F^c_{1,2} + F^p_{2,2} \cdot F^c_{2,2} + F^p_{3,2} \cdot F^c_{3,2}} = 
220 \cdot \frac{0.49 \cdot 0.984}{0.26 \cdot 0.004 + 0.25 \cdot 0.011 + 0.49 \cdot 0.984} = 218.28\\

V_{1} = 70.45 + 0.46 = 70.91\\
V_{2} = 441.72 + 1.26 = 442.98\\
V_{3} = 237.83 + 218.28 = 456.11\\
\end{array}
\end{equation}

\begin{verbatim}
[1] 70.44885
\end{verbatim}

\begin{verbatim}
[1] 441.7206
\end{verbatim}

\begin{verbatim}
[1] 237.8306
\end{verbatim}

\begin{verbatim}
[1] 0.4653881
\end{verbatim}

\begin{verbatim}
[1] 1.254918
\end{verbatim}

\begin{verbatim}
[1] 218.2797
\end{verbatim}

\begin{verbatim}
[1] 70.91
\end{verbatim}

\begin{verbatim}
[1] 442.98
\end{verbatim}

\begin{verbatim}
[1] 456.11
\end{verbatim}

Considering both population and cost allocation factors in \(V_{i}\),
the jobs at \(E1\) that are allocated to all population centers are
still preserved (i.e., \(V_{1,1} + V_{2,1} + V_{3,1} = O_1\)).
Additionally, the sum of jobs at \(E2\) are also all preserved (i.e.,
\(V_{1,2} + V_{2,2} + V_{3,2} = O_2\)). Thus the sum of \(V_{i}\) equals
the sum of opportunities (i.e., ) Notice that \(V_{i}\), allocates a
number of jobs to \(P_1\), \(P_2\), and \(P_3\) is between the values
allocated in \(V^p_{i}\) and \(V^c_{i}\).

When comparing \(V_i\) to the singly-constrained gravity model (see
Wilson \citeyearpar{wilson1971}), \(V_i\) is the result of constraining
\(A_i\) to match one of the marginals in the origin-destination table,
the known total of opportunities. Since the sum of opportunities is
preserved in the procedures above, it is possible to calculate an
interpretable measure of spatial availability per capita (lower-case
\(v_i\)) as shown in Equation (\ref{eq:SA-per-capita}).

\begin{equation}
\label{eq:SA-per-capita}
v_i = \frac{V_i}{P_i}
\end{equation}

To complete the illustrative example, the per capita spatial
availability of jobs is calculated in Equation
(\ref{eq:SA-per-capita-2populations}).

\begin{equation}
\label{eq:SA-per-capita-2populations}
\begin{array}{l}
v_{1} = \frac{V_{1,1} + V_{1,2}}{P_1} =  \frac{70.91}{260} = 0.272\\
v_{2} =  \frac{V_{2,1} + V_{2,2}}{P_2} =  \frac{442.98}{255} = 1.737\\
v_{3} =  \frac{V_{3,1} + V_{3,2}}{P_3} =  \frac{456.11}{495} = 0.921\\
\end{array}
\end{equation}

\begin{verbatim}
[1] 0.2727308
\end{verbatim}

\begin{verbatim}
[1] 1.737176
\end{verbatim}

\begin{verbatim}
[1] 0.9214343
\end{verbatim}

We can see that since \(P_2\) is closest to \(E1\), is similarly spaced
out from \(P1\) and \(P2\), and is a smaller population center thus
having less competition, \(P_2\) benefits with a higher spatial
availability of jobs per job-seeking population. We can also compare
these values to the overall ratio of jobs-to-population in this region
of two job center and three population centers is
\(\frac{750+220}{260+255+495}=\) 0.96 jobs per person.

\hypertarget{part-2-empirical-example-of-toronto}{%
\section{Part 2: Empirical example of
Toronto}\label{part-2-empirical-example-of-toronto}}

In this section we use population and employment data from the Golden
Horseshoe Area (GGH). This is the largest metropolitan region in Canada
and includes the cities of Toronto and Hamilton. We calculate gravity
accessibility, XXX, and the proposed spatial availability for Toronto
after introducing the data used and calibrating an impedance function.

\hypertarget{data}{%
\subsection{Data}\label{data}}

Population and employment data are drawn from the 2016 Transportation
Tomorrow Survey (TTS). This survey collects representative urban travel
information from 20 municipalities contained within the GGH area in the
southern part of Ontario, Canada (see Figure
\ref{fig:TTS-16-survey-area}) \citep{data_management_group_tts_2018}.
The data set includes Traffic Analysis Zones (TAZ) (n=3,764), the number
of jobs (n=3,081,885) and workers (n=3,446,957) at each origin and
destination. The TTS data is based on a representative sample of between
3\% to 5\% of households in the GGH and is weighted to reflect the
population covering the study area has a whole
\citep{data_management_group_tts_2018}.

To generate the travel cost for these trips, travel times between
origins and destinations are calculated for car travel using the R
package \{r5r\} \citep{r5r_2021} with a street network retrieved from
OpenStreetMap for the GGH area. A the 3 hr travel time threshold was
selected as it captures 99\% of population-employment pairs (see the
travel times summarized in Figure \ref{fig:TTS-16-survey-area}). This
method does not account for traffic congestion or modal split, which can
be estimated through other means
\citep[e.g.,][]{allen_suburbanization_2021, higgins2021changes}. For
simplicity, we carry on with the assumption that all trips are taken by
car in uncongested travel conditions.

All data and data preparation steps are documented and can be freely
explored in the companion open data product
\href{https://github.com/soukhova/TTS2016R}{\{TTS2016R\}}.

\begin{figure}

{\centering \includegraphics[width=0.8\linewidth]{images/TTS16-survey-area} 

}

\caption{\label{fig:TTS-16-survey-area}TTS 2016 study area (GGH, Ontario, Canada) along with the descriptive statistics of the trips, calculated origin-destination car travel time (TT), workers per TAZ, and jobs per TAZ. Contains 20 regions (black boundaries) and sub-regions (dark gray boundaries).}\label{fig:TTS-16-survey-area}
\end{figure}

\hypertarget{calibration-of-an-impedance-function}{%
\subsection{Calibration of an impedance
function}\label{calibration-of-an-impedance-function}}

In the synthetic example introduced in a preceding section, a negative
exponential function with an arbitrary parameter was used. For the
empirical example, we calibrate an impedance function on the trip length
distribution (TLD) of commute trips. Briefly, a TLD represents the
proportion of trips that are taken at a specific travel cost (e.g.,
travel time); this distribution is commonly used to derive impedance
functions in accessibility research
\citep{horbachov_theoretical_2018, batista_estimation_2019}.

The empirical and theoretical TLD for this data set are represented in
the top-left panel of Figure \ref{fig:TLD-Gamma-plot}. Maximum
likelihood estimation and the Nelder-Mead method for direct optimization
available within the \{fitdistrplus\} package \citep{fitdistrplus_2015}
were used. Based on goodness-of-fit criteria and diagnostics seen in
Figure \ref{fig:TLD-Gamma-plot}, the gamma distribution was selected
(also see Figure \ref{fig:plot-cullen-frey} in Appendix XX).

\begin{verbatim}
[1] 3069541
\end{verbatim}

\begin{figure}

{\centering \includegraphics[width=0.8\linewidth]{images/impedance_function} 

}

\caption{\label{fig:TLD-Gamma-plot}Car trip length distribution and calibrated gamma distribution impedance function (red line) with associated Q-Q and P-P plots. Based on TTS 2016.}\label{fig:TLD-Gamma-plot}
\end{figure}

The gamma distribution takes the following general form where the
estimated `shape' is \(\alpha=\) 2.019, the estimated `rate' is
\(\beta =\) 0.094, and \(\Gamma(\alpha)\) is defined in Equation
(\ref{gamma-dist}).

\begin{equation}
\label{gamma-dist}
\begin{array}{l} 
f(x, \alpha, \beta) = \frac {x^{\alpha-1}e^{-\frac{x}{\beta}}}{ \beta^{\alpha}\Gamma(\alpha)} \quad \text{for } 0 \leq x \leq \infty\\

\Gamma(\alpha) =  \int_{0}^{\infty} x^{\alpha-1}e^{-x} \,dx\\
\end{array}
\end{equation}

\hypertarget{measuring-access-to-jobs-in-toronto}{%
\subsection{Measuring access to jobs in
Toronto}\label{measuring-access-to-jobs-in-toronto}}

Toronto is the largest city in the GGH and represents a significant
subset of workers and jobs in the GGH; 31\% of workers in the GGH travel
to jobs in Toronto and 40\% of jobs are located within Toronto.

\begin{figure}
\includegraphics[width=1\linewidth]{Spatial-Availability-Refreshed_files/figure-latex/plot-access-SA-TO-1} \caption{\label{fig:plot-access-SA-TO}Calculated accessibility (top) and spatial availability (bottom) of employment from origins in destinations and origins in Toronto. Greyed out TAZ represent null accessibility and spatial availability values.}\label{fig:plot-access-SA-TO}
\end{figure}

To enhance the interpretability, spatial availability can be normalized
to provide more meaningful insight into how many jobs are
\emph{available} on average for each TAZ. This normalization, shown in
Figure \ref{fig:plot-avail-GGH-TTS-per-worker}, demonstrates which TAZ
have above (reds) and below (blue) the average available jobs per worker
in the GGH (1.17). Similar to the spatial availability plot of the GGH
jobs in Figure \ref{fig:plot-access-SA-GGH-TTS}, we can see that many
average or above average jobs per worker TAZ (whites and reds) are
present in southern Peel and Halton (south-west of Toronto), Waterloo
and Brantford (even more south-west of Toronto), and Hamilton and
Niagara (south of Toronto), however, the distribution is uneven and many
TAZ within these areas do have below average values (blues).

Interestingly, when considering \emph{competitive} job access, many
areas outside of Toronto have similar jobs per worker values as TAZ in
Toronto. This is contrary to the notion that since Toronto has high job
access it has a significant density of employment opportunities in the
GGH. Not all jobs in Toronto are \emph{available} since Toronto has a
high density of \emph{competition} in addition to density of jobs
opportunities. For instance, urban centers outside of Toronto such as
those found in Brantford, Guelph, southern Peel, Halton, and Niagara
have TAZ which are far above the the TTS average jobs per worker and
higher than TAZ within Toronto. High job access is not seen in the
accessibility plot which suggests that these less densely populated
urban centers may have sufficient employment opportunities for their
populations; this finding is obscured when only considering the
accessibility measure for job access as will be later discussed.

It is also worth noting that there is almost two times more jobs per
worker in the GGH jobs spatial availability results than the GGH Toronto
spatial availability results. This suggests that all GGH people who work
in the city of Toronto, on average, face more competition for jobs than
all GGH people who work anywhere in the GGH .

\begin{figure}
\includegraphics[width=1\linewidth]{Spatial-Availability-Refreshed_files/figure-latex/plot-avail-TO-per-worker-1} \caption{\label{fig:plot-avail-TO-per-worker}Spatial availability per worker, from origins to job opportunities in Toronto.}\label{fig:plot-avail-TO-per-worker}
\end{figure}

\newpage

\hypertarget{discussion-and-conclusions}{%
\section{Discussion and Conclusions}\label{discussion-and-conclusions}}

\hypertarget{appendix-a-step-by-step-accessibility-calculations-for-synthetic-example}{%
\section{Appendix A: Step-by-step accessibility calculations for
synthetic
example}\label{appendix-a-step-by-step-accessibility-calculations-for-synthetic-example}}

Details for the synthetic example:

\begin{table}

\caption{\label{tab:toy-example-table-appendix}\label{tab:toy-example}Summary description of synthetic example}
\centering
\begin{tabular}[t]{llrrr>{}l}
\toprule
Origin & Destination & Population & Jobs & Distance &  \\
\midrule
Population 1 & Employment Center 1 & 260 & 750 & 2548.1 & \\

Population 1 & Employment Center 2 & 260 & 220 & 5419.1 & \\

Population 2 & Employment Center 1 & 255 & 750 & 1314.1 & \\

Population 2 & Employment Center 2 & 255 & 220 & 4762.6 & \\

Population 4 & Employment Center 1 & 495 & 750 & 2170.2 & \\

Population 4 & Employment Center 2 & 495 & 220 & 1790.1 & \multirow{-6}{*}{\raggedright\arraybackslash \includegraphics{images/figure-1.png}}\\
\bottomrule
\end{tabular}
\end{table}

\noindent and: \[
\beta = 0.0015 \space in \space f(c_{ij}) = exp(\beta *distance_{ij})
\]

\hypertarget{conventional-gravity-accessibiliy}{%
\subsection{Conventional gravity
accessibiliy}\label{conventional-gravity-accessibiliy}}

\[
A_i = \sum_{j=1}^JO_j \cdot f(c_{ij})
\]

Solved in one step: \[
\sum_{j=1}^JO_j = E1 + E2 =  750 + 220 = 970 \space jobs
\]

\[
A_{P1} = 750 \cdot \exp(-0.0015 *2548.1) + 220 \cdot \exp(-0.0015 *5419.1) = 16.5 \\
A_{P2} = 750 \cdot \exp(-0.0015 *1314.1) + 220 \cdot \exp(-0.0015 *4762.6) = 104.7 \\
A_{P4} = 750 \cdot \exp(-0.0015 *2170.2) + 220 \cdot \exp(-0.0015 *1790.1) = 43.9
\]

\(A_{P1}\), \(A_{P2}\), and \(A_{P3}\) values represent the number of
travel-cost adjusted opportunities accessible to each population.
Specifically, only a proportion of opportunities are allocated to
population centers based on their travel cost value (higher the travel
cost lower the number of opportunities). The population is not
considered in this measure and the allocation of opportunities is not
constrained, it is only adjusted based on the weight of the travel cost.
With our negative exponential distance decay, accessibility can be as
high as 970 (the total number of opportunities in the region) and as low
as essentially 0.

However, in many instances being close to opportunities doesn't
necessarily mean much practically to an individual nor can this scale of
0 to the maximum number of total opportunities in the region be
operationalized by decision-makers. However, correlates have been found
(XX) so it is a strong indicator of urban structure, but practically
what does it mean for an individual to live in a population center of
\(A_{P4} =\) 43.9 jobs? On a scale of 0 to 970 (\(f(c_{ij})=0\) to
\(f(c_{ij})=1\)), this value is low but of the three population centers
it is around average. However, \(A_{P4}\) also has the largest
population of all population centers. It has a population that is less
than two times the population center of \(A_{P1}\) but an accessibility
value that is greater than two times \(A_{P1}\)'s accessibility value.
Does this mean that accessibility, after adjusting for population, is
greater than in \(P4\)? It is hard to say since the populations have
different travel costs to the opportunities. From this perspective,
competitive measures such as the FCA were introduced with the most
recently popularized 2SFCA (XX) discussed as follows.

\hypertarget{step-floating-catchment-approach-2sfca}{%
\subsection{2 step floating catchment approach
(2SFCA)}\label{step-floating-catchment-approach-2sfca}}

Step one:

\begin{equation}
\begin{array}{l}
R_{j} = \frac{O_{j}}{\sum_i P_{i} \cdot f(c_{ij})}\\

R_{E1} = \frac{750}{260 \cdot \exp(-0.0015 *2548.1) + 255 \cdot \exp(-0.0015 *1314.1) + 495 \cdot \exp(-0.0015 *2170.2)}\\
R_{E1} = 12.4 \space jobs \space per \space travel \space cost \space adjust. \space pop\\
R_{E2} = \frac{220}{260 \cdot \exp(-0.0015 *5419.1) + 255 \cdot \exp(-0.0015 *4762.6) + 495 \cdot \exp(-0.0015 *1790.1)}\\
R_{E2} = 6.5 \space jobs \space per \space travel \space cost \space adjust. \space pop\\
\end{array}
\end{equation}

Step two:

\begin{equation}
\begin{array}{l}
A_{i} = {\sum_j R_{j} \cdot f(c_{ij})}\\
A_{P1} = 12.4 \cdot \exp(-0.0015 *2548.1) + 6.46 \cdot \exp(-0.0015 *5419.1) = 0.27 \\
A_{P2} = 12.4 \cdot \exp(-0.0015 *1314.1) + 6.46 \cdot \exp(-0.0015 *4762.6) = 1.73 \\
A_{P4} = 12.4 \cdot \exp(-0.0015 *2170.2) + 6.46 \cdot \exp(-0.0015 *1790.1) = 0.92
\end{array}
\end{equation}

We see that the PPR \(R_{j}\) for each employment center can be
interpreted as the total number of jobs accessible to the total
travel-cost adjusted population. This step recognizes that not all
opportunities can be distributed to the entire population evenly since
not \emph{all} opportunities can be reached by \emph{all} population
centers. It is assumed that all population and employment centers are in
the same catchment. In step two, \(A_{i}\) values represent the
travel-cost adjusted PPR for each population center. Put another way,
here \(A_{P1}\), \(A_{P2}\), and \(A_{P3}\) values represent the number
of jobs accessible to each population center after being travel-cost
adjusted from both the opportunities-perspective and
population-perspective. The value could theoretically be on a scale of 0
to the maximum total number of PPR in the catchment (i.e.,
\(f(c_{ij})=0\) to \(f(c_{ij})=1\)); in this case that value is 18.9.

The method assumes unconstrained allocation based on travel cost thus
results can be interpreted in a similar way as the conventional gravity
based accessibility but from the perspective of PPR instead of
opportunities. This means that PPR is allocated based on the
\emph{potential} for interaction.

Looking at \(P4\), the largest and most central (to both employment
centers) population center, it has a 3.4 times greater 2SFCA value
compared to \(P1\) (conventional accessibility is only 2.6 times
greater). \(P4\) also has an 2SFCA value which is closer in the value to
\(P2\), it is only 0.53 times smaller than \(P2\) (conventional
accessibility is 0.42 times smaller). Since the 2SFCA adjusts
accessibility from both population-side and opportunity-side
travel-costs, small populations which are close to big employment
centers (\(P2\) close to \(E1\)) but also are in competition with
relatively close and large population centers (\(P4\)) have less
relative accessibility (\(P2\) vs.~\(P4\)) than conventional
accessibility. However, as discussed by \citet{paez2019}, the PPR
calculation in the first step and allocation of PPR to origins in the
second step is not \emph{proportional} to the total population seeking
opportunities. Though the `potential' for interaction is being
consistently allocated in these two steps, when looking to interpret the
measure from the perspective of allocation, the resulting values are
difficult to interpret. This issue of interpretability has been
attempted to be remedied by adjusting the population and opportunities
in both steps by a \emph{proportional} travel cost in the B2SFCA as
follows.

\hypertarget{balanced-2-step-floating-catchment-approach-b2sfca}{%
\subsection{balanced 2 step floating catchment approach
(B2SFCA)}\label{balanced-2-step-floating-catchment-approach-b2sfca}}

Step one:

\begin{equation}
\begin{array}{l}

R_{j} = \frac{O_{j}}{\sum_i P_{i} \frac{f(c_{ij})}{\sum_j f(c_{ij})}}\\

R_{E1} = \frac{750}{260 \frac{\exp(\beta *2548.1)}{\exp(\beta *2548.1) + \exp(\beta *5419.1)} + 255 \frac{\exp(\beta *1314.1)}{\exp(\beta *1314.1) + \exp(\beta *4762.6)} + 495 \frac{\exp(\beta *2170.2)}{\exp(\beta *2170.2) + \exp(\beta *1790.1)}}\\
R_{E2} = \frac{220}{260 \frac{\exp(\beta*5419.1)}{\exp(\beta *2548.1) + \exp(\beta *5419.1)} + 255 \frac{\exp(\beta*4762.6)}{\exp(\beta*1314.1) + \exp(\beta *4762.6)} + 495\frac{\exp(\beta*1790.1)}{\exp(\beta *2170.2) + \exp(\beta *1790.1)}}\\

R_{E1} = 1.09 \space jobs \space per \space proportional \space travel \space cost \space adjust. \space pop\\
R_{E2} = 0.68 \space jobs \space per \space proportional \space travel \space cost \space adjust. \space pop\\
\end{array}
\end{equation}

Step two:

\begin{equation}
\begin{array}{l}
A_{i} = {\sum_j R_{j}\frac{f(c_{ij})}{\sum_j f(c_{ij})}}\\A_{P1} = 1.09\frac{\exp(\beta*2548.1)}{\exp(\beta *2548.1) + \exp(\beta *1314.1) + \exp(\beta *2170.2)} + 0.68 \frac{\exp(\beta*5419.1)}{\exp(\beta *5419.1) + \exp(\beta *2170.2) + \exp(\beta *1790.1)} \\
A_{P2} = 1.09\frac{\exp(\beta *1314.1)}{\exp(\beta *2548.1) + \exp(\beta *1314.1) + \exp(\beta *2170.2)} + 0.68 \frac{\exp(\beta *4762.6)}{\exp(\beta *5419.1) + \exp(\beta *2170.2) + \exp(\beta *1790.1)} \\
A_{P4} = 1.09 \frac{\exp(\beta *2170.2)}{\exp(\beta *2548.1) + \exp(\beta *1314.1) + \exp(\beta *2170.2)} + 0.68 \frac{\exp(\beta *1790.1)}{\exp(\beta *5419.1) + \exp(\beta *2170.2) + \exp(\beta *1790.1)} \\
A_{P1} = 0.12 \\
A_{P2} = 0.77 \\
A_{P4} = 0.88 \\
\end{array}
\end{equation}

In the B2SFCA, the PPR \(R_{j}\) for each employment center can be
interpreted as the total number of jobs accessible to the total
population after being \emph{proportionally} adjusted to the travel
cost. The PPR \(R_{j}\) is then allocated, proportionally based on
travel cost, to each employment center. For this reason, the sum of all
\(A_{i}\) adds up to 1.77, the same value as the sum of all \(R_{j}\).

It should be also noted that when using the \emph{proportional} travel
costs to adjust opportunities and population, instead of using the raw
travel costs as in the 2SFCA, the resulting \(A_{i}\) values are
different. For instance, \(A_{P2}\) is the highest value (1.88 times her
than P4) in the 2SFCA but in the B2SFCA \(A_{P4}\) has the highest
value. In both 2SFCA and conventional accessibility, \(A_{P2}\) is the
highest value since P2 is the most populous population center and has
the closest proximity (low travel cost) to the most populous E1. Though
population size is considered in the first step of 2SFCA, the low
population size of P2 is not significant enough to result in a large
enough decrease in \(A_{P2}\) relative to \(A_{P4}\) (compared to
conventional accessibility in which the acccesibility values between P2
and P4 are relatively greater). However, in B2SFCA, the smaller
population size of P2 has a dramatic impact on the \(A_{P2}\) value.
This is because the travel-cost adjustment is proportional so the
proportionally large population size of P4 results in a smaller PPR for
both employment centers and thus a smaller amount of PPR from E1 is
allocated to P2 in step 2. Since PPR and the subsequent \(A_{i}\) are
proportionally allocated based travel costs, values no longer consider
\emph{potential} interaction and instead represent the allocation of
employment center PPR, based on travel time, to each population center.

This measure introduces some consistency in how the PPR is calculated
and is allocated to P1, P2, and P4, but is still lacking
interpretability in the resulting values.

\hypertarget{inverse-balancing-accessibility-doubly-constrained}{%
\subsection{Inverse Balancing accessibility,
doubly-constrained}\label{inverse-balancing-accessibility-doubly-constrained}}

\textbf{I don't get how to calculate this!!!}

This measure results in a opportunities per person metric, however, it
constraints opportunities and population from both sides, estimating
accessibility iteratively. The formulation requires the number of
opportunities equals the population so they propose the iterative
estimates are standardized such that opportunities equals population and
the disbalanced is caried through as a factor. The formulation for the
synthetic example would take the following form:

\begin{equation}
\begin{array}{l}
A_{i} = \frac{\bar A^{o}}{\bar A^{c}}{\sum_{j=1}^{J} \frac{O_{j}f(c_{ij})}{L_{j}}}\\
L_{i} = {\sum_{i=1}^{I} \frac{P_{i}f(c_{ij})}{A_{i}}}\\
\end{array}
\end{equation}

Iteration 1:

\begin{equation}
\begin{array}{l}
A_{P1} = (1)*\frac{750\exp(\beta*2548.1) + 220*\exp(\beta *5419.1)}{1} = 16.47\\
A_{P2} = (1)*\frac{750\exp(\beta*1314.1) + 220*\exp(\beta *4762.6)}{1} = 104.65\\
A_{P4} = (1)*\frac{750\exp(\beta*2170.2) + 220*\exp(\beta *1790.1)}{1} = 43.93\\
L_{E1} = \frac{260\exp(\beta*2548.1)}{16.47} + \frac{255*\exp(\beta *1314.1)}{104.65} + \frac{495*\exp(\beta *2170.2)}{43.93} = 1.12\\
L_{E2} = \frac{260\exp(\beta*5419.1)}{16.47} + \frac{255*\exp(\beta *4762.6)}{104.65} + \frac{495*\exp(\beta *1790.1)}{43.93} = 0.78\\
\end{array}
\end{equation}

We can complete \textasciitilde6 more iterations until we reach close to
convergence. I skip writing them out but the final \(A_{i}\) values
appear like:

\begin{equation}
\begin{array}{l}
A_{P1} = 14.6\\
A_{P2} = 92.4\\
A_{P4} = 47.0\\
\end{array}
\end{equation}

Here we see that \(A_{P1}\), \(A_{P2}\), and \(A_{P3}\) values represent
the number of opportunities, after being adjusted based on
opportunity-side and population-side travel costs, which are accessible
to each population. Compared to conventional accessibility, the
population is enche population is not considered in this measure and the
allocation of opportunities is not constrained, it is only adjusted
based on the weight of the travel cost. With our negative exponential
distance decay, accessibility can be as high as 970 (the total number of
opportunities in the region) and as low as essentially 0.

However, in many instances being close to opportunities doesn't
necessarily mean much practically to an individual nor can this scale of
0 to the maximum number of total opportunities in the region be
operationalized by decision-makers. However, correlates have been found
(XX) so it is a strong indicator of urban structure, but practically
what does it mean for an individual to live in a population center of
\(A_{P4} =\) 43.9 jobs? On a scale of 0 to 970 (\(f(c_{ij})=0\) to
\(f(c_{ij})=1\)), this value is low but of the three population centers
it is around average. However, \(A_{P4}\) also has the largest
population of all population centers. It has a population that is less
than two times the population center of \(A_{P1}\) but an accessibility
value that is greater than two times \(A_{P1}\)'s accessibility value.
Does this mean that accessibility, after adjusting for population, is
greater than in \(P4\)? It is hard to say since the populations have
different travel costs to the opportunities. From this perspective, the
FCA were introduced with the most recently popularized 2SFCA (XX)
discussed as follows.

\newpage

\renewcommand\refname{References}
\bibliography{bibliography.bib}


\end{document}
